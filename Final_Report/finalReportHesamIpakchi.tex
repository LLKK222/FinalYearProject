%%%%%%%%%%%%%%%%%%%%%%%%%%%%%%%%%%%%%%%%%
% Masters/Doctoral Thesis 
% LaTeX Template
% Version 1.42 (19/1/14)
%
% This template has been downloaded from:
% http://www.latextemplates.com
%
% Original authors:
% Steven Gunn 
% http://users.ecs.soton.ac.uk/srg/softwaretools/document/templates/
% and
% Sunil Patel
% http://www.sunilpatel.co.uk/thesis-template/
%
% License:
% CC BY-NC-SA 3.0 (http://creativecommons.org/licenses/by-nc-sa/3.0/)
%
% Note:
% Make sure to edit document variables in the Thesis.cls file
%
%%%%%%%%%%%%%%%%%%%%%%%%%%%%%%%%%%%%%%%%%

%----------------------------------------------------------------------------------------
%	PACKAGES AND OTHER DOCUMENT CONFIGURATIONS
%----------------------------------------------------------------------------------------

\documentclass[11pt, a4paper, oneside]{Thesis} % Paper size, default font size and one-sided paper

\graphicspath{{images/}} % Specifies the directory where pictures are stored

\hypersetup{urlcolor=black, colorlinks=true} % Colors hyperlinks in blue - change to black if annoying
\title{\ttitle} % Defines the thesis title - don't touch this

\begin{document}

\frontmatter % Use roman page numbering style (i, ii, iii, iv...) for the pre-content pages

\setstretch{1.3} % Line spacing of 1.3

\pagestyle{plain}

\newcommand{\HRule}{\rule{\linewidth}{0.5mm}} % New command to make the lines in the title page

% Subreferences Setup
\captionsetup{subrefformat=parens}

% PDF meta-data
\hypersetup{pdftitle={\ttitle}}
\hypersetup{pdfsubject=\subjectname}
\hypersetup{pdfauthor=\authornames}
\hypersetup{pdfkeywords=\keywordnames}

%----------------------------------------------------------------------------------------
%	DEFINITIONS OF NEW COMMANDS
%----------------------------------------------------------------------------------------

% vector variable command
\newcommand*\vecvar[1]{\mathbfit#1}

% Euclidean norm vector command
\newcommand{\norm}[1]{\left\lVert#1\right\rVert}

% inner-products vectors command
\newcommand{\innerP}[1]{\left\langle#1\right\rangle}

% matrix variable command
\newcommand*\matvar[1]{\mathbf#1}

% transpose command
\newcommand*\transpose[1]{#1^{T}}

% trace command
\newcommand*\trace[1]{tr(#1)}

% Bernoulli distribution command
\newcommand*\bernoulli[1]{Be(#1)}

% Normal distribution command
\newcommand*\normal[2]{\mathcal{N}(#1#2)}

% i.i.d. command
\newcommand*\iid{i.i.d.\xspace}

% probability event command
\newcommand*\probability[1]{\mathbb{P}(#1)}

% conditional 'given that' command
\newcommand*\givenbase[1][]{\:#1\lvert\:}
\let\given\givenbase

% expectation command
\newcommand*\expectation[1]{\mathbb{E}(#1)}

% variance command
\newcommand*\variance[1]{Var(#1)}

% covariance command
\newcommand*\covariance[2]{Cov(#1#2)}

% graph variable command
\newcommand*\graphvar[1]{\mathcal#1}

% set variable command
\newcommand*\setvar[1]{\mathcal#1}

% cardinality set command
\newcommand*\cardinality[1]{\left\vert#1\right\vert}

% mean command
\newcommand*\mean[1]{\langle#1\rangle}

% absolute value command
\newcommand*\abs[1]{\left\vert#1\right\vert}

% emphasise in text command
\newcommand*\emphT[1]{\textit{#1}}

% natural logarithm command
\newcommand*\natlog[1]{\log \left( #1 \right)}

% exponential function command
\let\oldexp\exp
\renewcommand{\exp}[1]{\oldexp(#1)}

% euler constant (e) command
\newcommand{\euler}{e\xspace}

% differentiate with repect to time command
\newcommand*\timeDiff[1]{\dot{#1}}

% set of real numbers 'R' command
\newcommand*\realsR{\mathbb{R}\xspace}

% special number one (1) command
\newcommand*\one{\mathds{1}\xspace}

% big O command
\newcommand*\bigO[1]{O\left(#1\right)}

% argmax command
\newcommand*{\argmax}{\operatornamewithlimits{argmax}\limits}

% argmin command
\newcommand*{\argmin}{\operatornamewithlimits{argmin}\limits}

%----------------------------------------------------------------------------------------
%	COVER PAGE
%----------------------------------------------------------------------------------------

\setboolean{@twoside}{false}
\includepdf[pages={-},offset=75 -75]{ReportCoverPage.pdf}

% blank page
\thispagestyle{empty}
\mbox{}
\newpage

%----------------------------------------------------------------------------------------
%	TITLE PAGE
%----------------------------------------------------------------------------------------

\begin{titlepage}
\begin{center}

\includegraphics[scale=0.1]{imperialCrest} % University/department logo - uncomment to place it 

\textsc{\LARGE \univname}\\[1cm] % University name
\textsc{\Large Final Year Project Report}\\[0.5cm] % Thesis type

\HRule \\[0.4cm] % Horizontal line
{\huge \bfseries \ttitle}\\[0.1cm] % Thesis title
\HRule \\[1.5cm] % Horizontal line
 
\begin{minipage}{0.4\textwidth}
\begin{flushleft} \large
\emph{Author:}\\
{\authornames} % Author name - remove the \href bracket to remove the link
\end{flushleft}
\end{minipage}
\begin{minipage}{0.4\textwidth}
\begin{flushright} \large
\emph{Supervisor:} \\
{\supname} \\% Supervisor name - remove the \href bracket to remove the link
\emph{Second marker:} \\
{\secmark}
\end{flushright}
\end{minipage}\\[2cm]
 
\large This report is submitted in fulfilment of the requirements\\ for the degree of \textit{\degreename}\\ % University requirement text
in the\\
\deptname\\\univname\\[1.5cm] % Research group name and department name

{\large \today}\\[0.1cm] % Date

\vfill
\end{center}

\end{titlepage}

% blank page
\thispagestyle{empty}
\mbox{}
\newpage

\setstretch{1.3} % Reset the line-spacing to 1.3 for body text (if it has changed)

%----------------------------------------------------------------------------------------
%	ABSTRACT PAGE
%----------------------------------------------------------------------------------------

\addtotoc{Abstract} % Add the "Abstract" page entry to the Contents

\abstract{
A very important area in network science is the detection of communities within networks, which helps to boost our understanding of many real-world complex systems.
Community detection is a very challenging problem, and a range of algorithms have been proposed, many based upon different approaches and techniques.
This project's aims are two-fold.
Firstly, we investigate and study different community detection algorithms, with a synthetic data-driven testing approach considered to develop a comprehensive comparison based on a range of network properties, using well-known statistical generative models.
Secondly, we fuse traditional community detection algorithms with analysis of correlation structure within the financial networks setting to investigate modified or tailored community detection algorithms that are applied to real-world and time-evolving financial data constructed from the daily prices of 80 stocks listed on the FTSE 100, during a period between 2004 and 2013.
The combination of our analysis using both synthetic and empirical data enables us to advocate the use of specific algorithms based upon the underlying network properties one wishes to detect communities from, in addition to validating techniques which may become useful for investors in their risk management of portfolios.
Furthermore, we are able to uncover interesting patterns in the evolving correlation structure of financial assets, which constitute a major equities exchange, during a significant period for economies worldwide.
}

\clearpage % Start a new page

%----------------------------------------------------------------------------------------
%	ACKNOWLEDGEMENTS
%----------------------------------------------------------------------------------------

\acknowledgements{
I would like to thank my supervisor Dr. Moez Draief for his support and advice during the project work in addition to his inspiring knowledge that has guided me throughout.

I would also like to thank my great friends and fellow colleagues who have made my studies the fantastic experience it has been.

Finally, I would like to thank my family for their everlasting support and love.
}

\clearpage % Start a new page

%----------------------------------------------------------------------------------------
%	LIST OF CONTENTS/FIGURES/TABLES PAGES
%----------------------------------------------------------------------------------------

\tableofcontents

\listoffigures

\listoftables


%----------------------------------------------------------------------------------------
%	NOTATIONS
%----------------------------------------------------------------------------------------

\clearpage % Start a new page

\listofnomenclature{ll}
{
$\cardinality{\setvar{S}}$ & Cardinality of the set $\setvar{S}$ \\
$\one_{\setvar{S}}$ & Indicator variable over the set $\setvar{S}$ \\
$y$ & Scalar $y$ \\
$\abs{y}$ & Absolute value of $y$ \\
$\vecvar{v}$ & Vector $\vecvar{v}$ \\
$\norm{\vecvar{v}}$ & Euclidean norm of the vector $\vecvar{v}$ \\
$\matvar{M}$ & Matrix $\matvar{M}$ \\
$M_{ij}$ & The element of the matrix $\matvar{M}$ at row $i$ and column $j$\\
$\transpose{\matvar{M}}$ & Transpose of the matrix $\matvar{M}$ \\
$\abs{\matvar{M}}$ & Determinant of the matrix $\matvar{M}$ \\
$\trace{\matvar{M}}$ & Trace of the matrix $\matvar{M}$ \\
$\expectation{X}$ & Expected value of $X$ \\
$\variance{X}$ & Variance of $X$ \\
$\exp{x}$ & Exponential function \\
$\natlog{x}$ & \emphT{natural logarithm} of $x$ (logarithm to the base $\euler$)
}

%----------------------------------------------------------------------------------------
%	THESIS CONTENT - CHAPTERS
%----------------------------------------------------------------------------------------

\mainmatter % Begin numeric (1,2,3...) page numbering

\nocite{*}

% Introduction
% Chapter 1

\chapter{Introduction}

\label{cha:introduction}

%----------------------------------------------------------------------------------------

Networks have been studied extensively to model many interesting complex systems, including the Internet, social networks, financial networks and biological networks \cite{New06a,DKM+13,MG13}.
Any network consists of \emphT{nodes} which represent items of interest, and \emphT{edges} which represent the connectivity between pairs of nodes. For example, considering social networks, nodes are the users and the edges correspond to interactions between the users.
An interesting feature many networks exhibit is \emphT{community structure}, which involves the natural dividing of nodes into groups, called \emphT{communities}, where there are denser connections within a group, and sparser connections between different groups \cite{New06a,DKM+13,For10,New06b}. This particular type of community structure is also known as \emphT{assortative} \cite{DKM+13}. For instance, social networks contain communities corresponding to real-life communities consisting of the members, such as friendship or family circles.
The problem of detecting communities within networks is known as \emphT{community detection}, and algorithms are developed as a solution.

In order to provide a theoretical setting to test and compare different community detection algorithms, generative models of random graphs are very useful, and one such commonly used model is the \emphT{stochastic block model} \cite{DKM+13,NN12}. We will investigate several community detection algorithms, and will use different generative models in order to analyse and reason about them.

The underlying ingredients of the community detection algorithms have other interesting applications also, including the analysis of time series data within the context of financial networks. The aim involves seeking groups of correlated financial assets that can be used in \emphT{mean variance portfolio optimisation}.
We consider the application of community detection algorithms to real-world financial networks, in order find groups of correlated stocks found on the FTSE 100 exchange.

This project aims is two fold; firstly, we investigate and study different community detection algorithms, and secondly, apply modifications to these techniques in order to use them detect communities in the financial networks setting.


% Background
% Chapter 2

\chapter{Background}

\label{cha:background}

%----------------------------------------------------------------------------------------

In this chapter we will describe all the technical background required to understand and detail the different settings we investigate.
Initially, we will highlight some basic results in graph theory.
Then, we will outline the problem of community detection and present statistical models used to generate random graphs with community structure to be used as a testing playground for algorithms.
Following this, we will discuss basic concepts within finance required to understand the behaviour of financial assets that will provide the motivation for applying community detection algorithms to financial networks.

%-----------------------------------------------------
%   Preliminaries in Networks Background Section
%-----------------------------------------------------

\section{Graph Theory Preliminaries}
\label{sec:graphTheoryBackground}

We assume the reader is familiar with some basic concepts in linear algebra such as matrix multiplication, eigenvectors and eigenvalues of matrices. Rather, we will cover some basic tools within spectral graph theory using definitions from \cite{For10,New06a, Spi12, Spi07}. Spectral graph theory is the study of graphs through the eigenvalues and eigenvectors of matrices associated with the graphs \cite{Spi12}. We begin by defining some basic notions about graphs.
\begin{definition}
	\label{def:graph}
	A \emphT{graph} $\graphvar{G}$ is a pair of sets (V,E), where V is a set of \emphT{vertices} or \emphT{nodes} and $E \subset V^{2}$, the set of unordered pairs of elements of V. The elements of E are called \emphT{edges} or \emphT{links}.
\end{definition}
\begin{definition}
	\label{def:undirectedGraph}
	A graph $\graphvar{G} = (V,E)$ is called \emphT{undirected} if for all $v,w \in V$: $(v,w) \in E \iff (w,v) \in E$. Otherwise, $\graphvar{G}$ is called \emphT{directed}.
\end{definition}
\begin{definition}
	\label{def:weightedGraph}
	A \emphT{weighted} graph is a graph where a number (weight) is assigned to each edge.
\end{definition}
We will assume, without loss of generality, that $V = \{1,\dots,n\}$. See \cref{fig:exampleGraphSmall} for an example of an undirected graph with seven vertices and eleven edges.
\begin{definition}
	\label{def:subGraph}
	A graph $\graphvar{G}^{\prime} = (V^{\prime},E^{\prime})$ is a \emphT{subgraph} of $\graphvar{G} = (V,E)$ if $V^{\prime} \subset V$ and $E^{\prime} \subset E$. If $\graphvar{G}^{\prime}$ contains all edges of $\graphvar{G}$ that join vertices of $V^{\prime}$, one says that the subgraph $\graphvar{G}^{\prime}$ is \emphT{induced} or \emphT{spanned} by $V^{\prime}$.
\end{definition}
\begin{definition}
	\label{def:cuts}
	A partition of the vertex set V in two subsets S and $V-S$ is called a \emphT{cut}. The \emphT{cut size} is the number of edges of $\graphvar{G}$ joining vertices of S with vertices of $V-S$.
\end{definition}
\begin{definition}
	\label{def:neighbourhoodNode}
	Two vertices are \emphT{adjacent} or \emphT{neighbours} if they are connected by an edge. The set of neighbours of a vertex $v$ is called \emphT{neighbourhood}, and denoted by $\Gamma(v)$.
\end{definition}
\begin{definition}
	\label{def:degreeNode}
	The \emphT{degree} $d_{v}$ of a vertex $v$ is the number of its neighbours, $\abs{\Gamma(v)}$.
\end{definition}
We will be interested in using certain graphs in the models, such as bipartite graphs.
\begin{definition}
	\label{def:bipartiteGraph}
	A \emphT{bipartite} graph, is a graph whose vertices can be decomposed into two disjoint sets such that no two vertices within the same set are adjacent.
\end{definition}
\begin{definition}
	\label{def:clique}
	A \emphT{clique} of an undirected graph is a subset of its vertices such that every pair of vertices in the subset are adjacent in the graph.
\end{definition}
An example of an undirected bipartite graph with nine vertices and eight edges is shown in \cref{fig:exampleGraphBipartite}, whilst an example of a clique within an undirected graph is shown in \cref{fig:exampleGraphClique}.

%---   FIGURE
\begin{figure}
\centering
	\begin{subfigure}{.5\textwidth}
		\centering
		\includegraphics[width=0.6\linewidth]{figures/exampleGraphSmall.png}
		\caption{}
		\label{fig:exampleGraphSmall}
	\end{subfigure}%
	\begin{subfigure}{.5\textwidth}
		\centering
		\includegraphics[width=0.6\linewidth]{figures/exampleGraphBipartite.png}
		\caption{}
		\label{fig:exampleGraphBipartite}
	\end{subfigure}\\
	\begin{subfigure}{.5\textwidth}
		\centering
		\includegraphics[width=0.6\linewidth]{figures/exampleGraphClique.png}
		\caption{}
		\label{fig:exampleGraphClique}
	\end{subfigure}
	\caption[Visualisations of example undirected graphs.]{\label{fig:exampleGraphs} A set of visualisations of undirected graphs. In \subref{fig:exampleGraphSmall}, the graph has seven nodes and eleven edges. In \subref{fig:exampleGraphBipartite}, a bipartite graph, with nine nodes (elements of disjoint sets are coloured black and red denoting membership) and eight edges, is shown. In \subref{fig:exampleGraphClique}, an undirected graph, with six nodes and six edges is shown, where the nodes coloured red form a clique within the graph.}
\end{figure}

There is a very close connection between graphs and matrices, since the whole information about the topology of a graph can be entailed in matrix form.
\begin{definition}
	\label{def:adjacencyMatrix}
	The \emphT{adjacency matrix}, $\matvar{A} \in \{0,1\}^{n \times n}$ of a graph $\graphvar{G} = (V,E)$, is a $n\times n$ matrix whose element $A_{ij}$ equals 1 if there exists an edge joining vertices i and j in $\graphvar{G}$, and zero otherwise.
\end{definition}
From \cref{def:adjacencyMatrix}, it follows that elements of the adjacency matrix, $\matvar{A}$, can be written as
\begin{equation}
	A_{ij} =
	\begin{cases}
		1 & \text{if } (i,j) \in E\\
		0 & \text{otherwise}
	\end{cases}
\end{equation}
Note that the sum of elements of the $i$-th row of the adjacency matrix yields the degree of node $i$ of the graph, $d_{i} = \sum_{j} A_{ij}$. Also, the adjacency matrix is symmetric if the graph is undirected.
\begin{definition}
	\label{def:weightedAdjacencyMatrix}
	The \emphT{weighted adjacency matrix}, $\matvar{A} \in \realsR^{n \times n}$ of a weighted graph $\graphvar{G} = (V,E)$, is a $n\times n$ matrix whose element $A_{ij}$ equals the weight of the edge connecting nodes $i$ and $j$, if it exists, and zero otherwise. 
\end{definition}

There are other matrices that have also been studied extensively in spectral graph theory, including the Laplacian which is applied in topics such as graph partitioning, synchronisation and graph connectivity \cite{For10}.
\begin{definition}
	\label{def:degreeMatrix}
	The \emphT{degree matrix}, $\matvar{D}$, of a graph $\graphvar{G} = (V,E)$, is a $n \times n$ diagonal matrix whose element $D_{ii}$ equals the degree of vertex $i$.
\end{definition}
From \cref{def:degreeMatrix}, it follows that elements of the degree matrix, $\matvar{D}$, can be written as
\begin{equation}
	 D_{ij} =
	\begin{cases}
		d_{i} & \text{if } i = j\\
		0 & \text{otherwise}
	\end{cases}
\end{equation}
\begin{definition}
	\label{def:unnormalisedLaplacianMatrix}
	The matrix $\matvar{L} = \matvar{D}  - \matvar{A} $ is called the \emphT{unnormalised Laplacian matrix}.
\end{definition}
From \cref{def:unnormalisedLaplacianMatrix}, it follows that elements of the unnormalised Laplacian matrix of a graph $\graphvar{G} = (V,E)$, $\matvar{L}$, can be written as
\begin{equation}
	L_{ij} =
	\begin{cases}
		d_{i} & \text{if } i = j\\
		-1 & \text{if } i \neq j \text{ and }  (i,j) \in E\\
		0 & \text{otherwise}
	\end{cases}
\end{equation}
\begin{definition}
	\label{def:normalisedLaplacianMatrix}
	The matrix $\widetilde{\matvar{L}} = \matvar{I}  - \matvar{D}^{-1/2}\matvar{A}\matvar{D}^{-1/2}$ is called the \emphT{normalised Laplacian matrix}, where $\matvar{I}$ is the $n \times n$ identity matrix.
\end{definition}
Note that from \cref{def:unnormalisedLaplacianMatrix,def:normalisedLaplacianMatrix}, the normalised Laplacian matrix can also be written as $\widetilde{\matvar{L}} = \matvar{D}^{-1/2}\matvar{L}\matvar{D}^{-1/2}$.

An important property of matrices is their spectra, which we shall analyse for certain matrices later in the report to motivate community detection algorithms.
\begin{definition}
	\label{def:spectrum}
	The \emphT{spectrum} of a graph $\graphvar{G}$ is the set of eigenvalues of its adjacency matrix, $\{\lambda_{1},\dots,\lambda_{n}\}$.
\end{definition}
\begin{definition}
	\label{def:spectralRadius}
	Let $\lambda_{1},\dots,\lambda_{n}$ be the eigenvalues of a matrix $\matvar{M} \in \realsR^{n \times n}$. The \emphT{spectral radius} is defined as $\rho(\matvar{M}) = \max\limits_{i} \abs{\lambda_{i}}$.
\end{definition}


%-----------------------------------------------------
%   Community Structure in Networks Background Section
%-----------------------------------------------------

\section{Community Structure in Networks}
\label{sec:communityStructureBackground}

An intuitive notion of communities within graphs involves the assignment of nodes to communities such that there are denser connections between nodes belonging to the same community, and sparser connections between nodes belonging to different communities.
If a graph exhibits this property, it is said to contain assortative community structure \cite{New06a,DKM+13,For10,New06b}.
For instance, within social networks, where nodes are users and edges between nodes represent interactions between the users, community structure within the graph corresponds to real-life communities consisting of the users and friendship circles.
The aim of community detection algorithms is to estimate or recover the node assignments.
The algorithms need to be efficient due to the moderate size of graphs realised in many real-world applications, so we require the computationally complexity to not be worse than nearly linear in the number of edges in the graph (approximately $\bigO{n^{2}\natlog{n}}$ where $n$ represents the number of nodes in the graph).
It is crucial to note that we are seeking to identify communities in graphs of a moderate size, and not very large scale graphs as are now increasingly studied in research for social networks applications (e.g. Facebook, Twitter, Google networks).
The reason for this decision lies in the very different approaches taken by researches for large scale applications than for small or moderate size networks.

Within the literature, the terms \emphT{groups} and \emphT{clusters} are synonymous with communities, and as such we will use all three terms interchangeably through the report; so the reader should note all these terms refer to the same notion of communities in graphs.
Moreover we also refer to the process of estimating node assignments as \emphT{partitioning} the network.

In order to help provide a setting where different algorithms may be compared, we wish to study particular models which generate random graphs.
One popular model is called the stochastic block model.
Many special cases of this model have been studied, but we consider two versions, both widely used in the literature.
Firstly, there is a model considered by Decelle et al. \cite{DKM+13} and Nadakuditi et al. \cite{NN12}, also known as the \emphT{planted partition model}.
Secondly, there is a model used by Montanari \cite{DM13,Mon13}, which we will refer to as the \emphT{hidden clique model}. 
We emphasise that we do not exclusively focus on detecting cliques for the latter model, but the name is simply convenient for reference in this report.

Let us define the stochastic block model following Decelle et al. \cite{DKM+13}.
The stochastic block model has parameters: $k$, $n_{a}$, $\matvar{P}$. $k$ represents the number of communities (or groups), $n_{a}$ refers to the expected fraction of nodes within each group $a$, for $1 \leq a \leq k$, and $\matvar{P}$ is a $k \times k$ matrix whose element $P_{ab}$ equals the probability of an edge occurring between nodes belonging to groups $a$ and $b$. It is known as the \emphT{affinity matrix}.
We proceed to generate a random directed graph, $\graphvar{G}$, consisting of $n$ nodes.
Firstly, though, assign, to each node $i$ of the graph, $\sigma_{i} \in \{1,\dots,k\}$, a label indicating which community the node belongs to.
These labels are chose independently, where, for each node $i$, $\probability{\sigma_{i} = a} = n_{a}$.
Let $\vecvar{u} = \transpose{[\sigma_{1},\dots,\sigma_{n}]}$ be the \emphT{ground-truth node assignments} of the graph.
The random graph is generated to have an adjacency matrix, $\matvar{A}$, whose elements are defined by
\begin{equation}
	\label{def:sbmAdjacencyMatrix}
	A_{ij} =
	\begin{cases}
		0 & \text{if } i = j\\
		X & \text{otherwise}
	\end{cases}
\end{equation}
where $X \sim \bernoulli{P_{\sigma_{i},\sigma_{j}}}$.

This formulation matches the intuition of community structure, that the connectivity between two nodes depends solely on the community memberships of the two nodes.
Note, also, that we do not allow self-loops.

The framework for testing community detection algorithms, which we will use, can now be summarised.
Firstly, we generate synthetic datasets, by creating random graphs from the models described in \cref{subsec:plantedPartitionModel,subsec:hiddenCliqueModel}, with varying parameters and known underlying ground-truth node assignments.
Then we use the synthetically-generated graphs as input to various algorithms (an appropriate model is chosen for each algorithm), which provides, as output, an estimate to the community assignments.
Finally, for each algorithm, we compare the estimated community assignments to the ground-truth values.
This provides a notion of performance and accuracy to compare between the algorithms.

We now describe two models; one is a special case of the stochastic block model, created by imposing specific properties on the parameters, and the other a slightly modified version.

We also stress to the reader that we are only considering \emphT{non-overlapping communities}, where each node may only belong to one particular community.

%-----------------------------------
%   Planted Partition Model Sub Section
%-----------------------------------

\subsection{Planted Partition Model}
\label{subsec:plantedPartitionModel}

We will consider the formulation of the planted partition model as given by Decelle et al. \cite{DKM+13}.
To construct the planted partition model, we consider the stochastic block model, with $n_{a} = 1/k$, and the affinity matrix defined by
\begin{equation}
	\label{def:ppmAffinityMatrix}
	P_{ab} =
	\begin{cases}
		p_{in} & \text{if } a = b\\
		p_{out} & \text{otherwise}
	\end{cases}
\end{equation}
These properties essentially result approximately equal number of nodes and edges within each community.
From this definition, $p_{in}$ represents the probability of an edge occurring between two nodes belonging to the same community and $p_{out}$ represents the probability of an edge occurring between two nodes belonging to the different communities.
From now on, we refer to $p_{in}$ and $p_{out}$ as the \emphT{edge occurrence probabilities} for this model.
We assume assortative structure so that $p_{in} > p_{out}$. This just matches the intuition of edges more likely to appear between nodes belonging to the same community than between nodes belonging to different communities.

An example of a random graph generated by the planted partition model is shown in \cref{fig:ppmUnlabelledAdjacencyMatrixPlot}.
We labelled nodes using $\sigma_{i} = 1 + (i \bmod{k})$ for $i = 1,\dots,n$ and generated the graph with $n = 300$, $k = 3$, $p_{in} = 0.7$, $p_{out} = 0.3$.
The adjacency matrix of this graph is plotted with a pixel shaded red if the element in the adjacency matrix, corresponding to the location of the pixel, equals 1; while a pixel is shaded white if the element equals 0.
Since we know the ground truth labelling of nodes, we can, without loss of generality, reorder the rows and columns of the adjacency matrix, such that it consists of blocks of nodes associated with the node community memberships.
This is plotted in \cref{fig:ppmLabelledAdjacencyMatrixPlot}. Note that since $k = 3$, there are $3 \times 3 = 9$ blocks, where the blocks are denser along the main diagonal since these correspond to edges between nodes belonging to the same community and $p_{in} > p_{out}$.

%---   FIGURE
\begin{figure}
	\centering
	\begin{subfigure}{.5\textwidth}
		\centering
		\includegraphics[width=0.8\linewidth]{figures/ppmAdjacencyMatrix.png}
		\caption{}
		\label{fig:ppmUnlabelledAdjacencyMatrixPlot}
	\end{subfigure}%
	\begin{subfigure}{.5\textwidth}
		\centering
		\includegraphics[width=0.8\linewidth]{figures/ppmLabelledAdjacencyMatrix.png}
		\caption{}
		\label{fig:ppmLabelledAdjacencyMatrixPlot}
	\end{subfigure}
	\caption[Plots of adjacency matrices of graph generated by planted partition model.]{\label{fig:ppmAdjacencyMatricesPlots} A set of plots for unlabelled, \subref{fig:ppmUnlabelledAdjacencyMatrixPlot}, and labelled, \subref{fig:ppmLabelledAdjacencyMatrixPlot}, adjacency matrices for random graph generated by planted partition model. The graphs were generated with $n = 300$, $k = 3$, $p_{in} = 0.7$ and $p_{out} = 0.3$.}
\end{figure}

In \cref{fig:ppmExampleGraph}, we show a visualisation of an instance of a random graph generated by the planted partition model with parameters with $n = 240$, $k = 3$, $p_{in} = 0.2$, $p_{out} = 0.01$.

%---   FIGURE
\begin{figure}
	\centering
	\includegraphics[width=0.6\linewidth]{figures/ppmExampleGraph.png}
	\caption[Visualisation of a graph generated by the planted partition model.]{\label{fig:ppmExampleGraph} A visualisation of an instance of a random graph generated by the planted partition model with $n = 240$, $k = 3$, $p_{in} = 0.2$ and $p_{out} = 0.01$.}
\end{figure}

Decelle et al. \cite{DKM+11} conjectured a phase transition for sparse graphs generated from the planted partition model, using non-rigorous ideas from statistical physics \cite{MNS12}.
Nadakuditi et al. \cite{NN12} used methods from random matrix theory to present an asymptotic analysis of spectra of random graphs to also demonstrate the presence of a phase transition.
Essentially, we can distinguish between a \textit{detectable} phase where it is possible to learn node assignments in a way that is correlated with the ground-truth node assignments of the graph, and an \textit{undetectable} phase, where learning is impossible.

Let us define, for convenience, the variables $c_{in} = np_{in}$ and $c_{out} = np_{out}$.
Consider a graph, generated by the planted partition model, and following the argument of \cite{NN12}, which we will not explain, one finds a transition occurring at the point
\begin{equation}
\label{eq:ppmPhaseTransitionK}
	c_{in} - c_{out} = \sqrt{k(c_{in} + (k-1)c_{out})}.
\end{equation}
In particular, let us consider the case where $k = 2$, so we find a transition at
\begin{equation}
\label{eq:ppmPhaseTransitionK=2}
	c_{in} - c_{out} = \sqrt{2(c_{in} + c_{out})}.
\end{equation}
Mossel et al. \cite{MNS12} proved the undetectable phase region of the conjecture given by equation \cref{eq:ppmPhaseTransitionK=2}.
That is to say, it is impossible to meaningfully recover the node assignments when $ c_{in} - c_{out} < \sqrt{2(c_{in} + c_{out})}$.
Massouli\'e \cite{Mas13} and then, independently using a different proof, Mossel et al. \cite{MNS13b} proved the detectable phase region of the conjecture, meaning it is possible to recover node assignments positively correlated with the ground-truth when $ c_{in} - c_{out} > \sqrt{2(c_{in} + c_{out})}$.
The techniques used to prove these results are beyond the scope of this report, however these results provide a very important limit on the ability of algorithms to detect communities (for the case of two underlying ground-truth communities).
This motivates the development of algorithms which can efficiently (in nearly linear time) detect communities, in the sparse regime, up to this limit.

%--------------------------------
%   Hidden Clique Model Sub Section
%--------------------------------

\subsection{Hidden Clique Model}
\label{subsec:hiddenCliqueModel}

We will consider the following formulation as explained by Montanari \cite{DM13,Mon13}.
To construct the hidden clique model, we consider the stochastic block model with some modifications.
We proceed to generate a graph with $n$ nodes and $k$ communities but with the affinity matrix, $\matvar{P}$, becoming a $(k+1) \times (k+1)$ matrix defined by
\begin{equation}
	\label{def:hcmAffinityMatrix}
	P_{ab} =
	\begin{cases}
		p_{in} & \text{if } a = b \text{, } a \leq k \text{, } b \leq k\\
		p_{out} & \text{otherwise}
	\end{cases}
\end{equation}
Another tweak is that we now consider the variable $n_{a}$ to represent the number of nodes within community $a$ (rather than the expected fraction of nodes). Note that the number of nodes within each community does not necessarily sum to $n$, since we consider them to be `hidden' within the graph.
Also, we are interested in the regime where the size of these communities is small relative to the size of the graph.
Once more, we assume assortative structure within the hidden communities so that $p_{in} > p_{out}$.

An example of a random graph generated by the hidden clique model is shown in \cref{fig:hcmUnlabelledAdjacencyMatrixPlot}.
We generated the graph with $n = 300$, $k = 3$, $p_{in} = 0.8$, $p_{out} = 0.2$, $n_{1} = 50$, $n_{2} = 40$, $n_{3} = 20$.
The adjacency matrix of this graph is plotted with a pixel shaded red if the element in the adjacency matrix, corresponding to the location of the pixel, equals 1; while a pixel is shaded white if the element equals 0.
Since we know the ground truth labelling of nodes, we can, without loss of generality, reorder the rows and columns of the adjacency matrix, such that it consists of blocks of nodes associated with the node community memberships.
This is plotted in \cref{fig:hcmLabelledAdjacencyMatrixPlot}. We can see three dense blocks of size $50$, $40$ and $20$ respectively, corresponding to the three hidden communities.

%---   FIGURE
\begin{figure}
	\centering
	\begin{subfigure}{.5\textwidth}
		\centering
		\includegraphics[width=0.8\linewidth]{figures/hcmAdjacencyMatrix.png}
		\caption{}
		\label{fig:hcmUnlabelledAdjacencyMatrixPlot}
	\end{subfigure}%
	\begin{subfigure}{.5\textwidth}
		\centering
		\includegraphics[width=0.8\linewidth]{figures/hcmLabelledAdjacencyMatrix.png}
		\caption{}
		\label{fig:hcmLabelledAdjacencyMatrixPlot}
	\end{subfigure}
	\caption[Plots of adjacency matrices of graph generated by hidden clique model.]{\label{fig:hcmAdjacencyMatricesPlots} A set of plots for unlabelled, \subref{fig:hcmUnlabelledAdjacencyMatrixPlot}, and labelled, \subref{fig:hcmLabelledAdjacencyMatrixPlot}, adjacency matrices for random graph generated by hidden clique model. The graphs were generated with $n = 300$, $k = 3$, $p_{in} = 0.8$, $p_{out} = 0.2$, $n_{1} = 50$, $n_{2} = 40$, $n_{3} = 20$.}
\end{figure}

In \cref{fig:hcmExampleGraph}, we show a visualisation of an instance of a random graph generated by the hidden clique model with parameters with $n = 150$, $k = 1$, $n_{1} = 30$, $p_{in} = 0.3$, $p_{out} = 0.05$.

%---   FIGURE
\begin{figure}
	\centering
	\includegraphics[width=0.6\linewidth]{figures/hcmExampleGraph.png}
	\caption[Visualisation of a graph generated by the hidden clique model.]{\label{fig:hcmExampleGraph} A visualisation of an instance of a random graph generated by the hidden clique model with $n = 150$, $k = 1$, $n_{1} = 30$, $p_{in} = 1.0$, $p_{out} = 0.1$. Nodes belonging to the hidden community, which in this case is in fact a clique, are coloured red whilst other nodes are coloured blue.}
\end{figure}

An interesting phase transition can also be derived for these models also. We follow the argument of Montanari \cite{DM13,Mon13} to show this. We consider the simplest case of the model with only one hidden community (i.e. $k = 1$).
We begin by generating a graph from the hidden clique model with $n$ nodes and one community.
Assume the $n_{1}$ nodes belonging to the hidden community make up a hidden community set, $\setvar{S}$ (so that $\cardinality{\setvar{S}} = n_{1}$). Define $\one_{n} \in \{1\}^{n}$ as the $n$-dimensional vector with every element equal to 1.
Also let $\one_{\setvar{S}}$ be the indicator variable for nodes belonging to the hidden community set.
Denote the adjacency matrix of the graph by $\matvar{A}$, where each element is defined by
\begin{equation}
	\label{def:hcmAdjacencyMatrix}
	A_{ij} \sim \bernoulli{p_{ij}}
\end{equation}
where 
\begin{equation}
	\label{def:hcmPijAdjacencyMatrix}
	p_{ij} =	
	\begin{cases}
		p_{in} & \text{if } i \in \setvar{S} \text{, } j \in \setvar{S} \\
		p_{out} & \text{otherwise}
	\end{cases}
\end{equation}
Then, we get the following
\begin{equation}
	\label{def:hcmExpectationAdjacencyMatrix}
	\expectation{\matvar{A}} = (p_{in} - p_{out})\one_{\setvar{S}}\transpose{\one_{\setvar{S}}} + p_{out}\one_{n}\transpose{\one_{n}}
\end{equation}
and
\begin{equation}
	\label{def:hcmVarianceAdjacencyMatrix}
	\variance{A_{ij}} = p_{out}(1-p_{out}) \text{ if } \{i,j\} \not\subseteq \setvar{S}
\end{equation}
Denote $\widetilde{\matvar{A}}$ as the \emphT{normalised adjacency matrix} of the graph, defined by
\begin{equation}
	\label{def:hcmNormlaisedAdjacencyMatrix}
	\widetilde{\matvar{A}} \equiv \frac{1}{\sqrt{np_{out}(1-p_{out})}} (\matvar{A} - p_{out}\one_{n}\transpose{\one_{n}})
\end{equation}
By taking the expectation and using \cref{def:hcmExpectationAdjacencyMatrix}, we obtain
\begin{equation}
	\label{def:hcmExpectationNormalisedAdjacencyMatrix}
	\expectation{\widetilde{\matvar{A}}} = \frac{1}{\sqrt{np_{out}(1-p_{out})}} (p_{in} - p_{out}) \one_{\setvar{S}}\transpose{\one_{\setvar{S}}}
\end{equation}
Let us write $\widetilde{\matvar{A}} = \expectation{\widetilde{\matvar{A}}} + (\widetilde{\matvar{A}} - \expectation{\widetilde{\matvar{A}}})$. Now using \cref{def:hcmExpectationNormalisedAdjacencyMatrix}, we get
\begin{equation}
	\label{def:hcmNormlaisedAdjacencyMatrix2}
	\widetilde{\matvar{A}} = \frac{1}{\sqrt{np_{out}(1-p_{out})}} (p_{in} - p_{out}) \one_{\setvar{S}}\transpose{\one_{\setvar{S}}} + (\widetilde{\matvar{A}} - \expectation{\widetilde{\matvar{A}}})
\end{equation}
Let us now define the following
\begin{equation}
	\label{def:hcmLambda}
	\lambda \equiv \frac{p_{in} - p_{out}}{\sqrt{np_{out}(1-p_{out})}}
\end{equation}
\begin{equation}
	\label{def:hcmU}
	\vecvar{u} \equiv \one_{\setvar{S}}
\end{equation}
\begin{equation}
	\label{def:hcmZ}
	\matvar{Z} \equiv \widetilde{\matvar{A}} - \expectation{\widetilde{\matvar{A}}}
\end{equation}
Then we can re-write \cref{def:hcmNormlaisedAdjacencyMatrix2} as
\begin{equation}
	\label{def:hcmNormlaisedAdjacencyMatrixDecomposed}
	\widetilde{\matvar{A}} = \lambda \vecvar{u}\transpose{\vecvar{u}} + \matvar{Z}
\end{equation}

One can interpret $\lambda$ as a signal-to-noise ratio, $\vecvar{u}$ as a signal (i.e. the ground-truth node assignments we which to infer) and $\matvar{Z}$ as zero-mean noise with \iid entries.
We have essentially represented the problem of inferring the hidden community from the graph by a problem of estimating a rank-1 matrix in noise. Notice that for the generalisation with $k$ communities, we would get a rank-k matrix plus noise for the normalised adjacency matrix.

Assume we generate a network associated with a normalised adjacency matrix, $\widetilde{\matvar{A}}$, defined in \cref{def:hcmNormlaisedAdjacencyMatrixDecomposed} and are interested in reconstructing the vector of node assignments, $\vecvar{u}$.
This problem has been investigated in many application under the guise of `Low-rank deformation of Wigner matrices' \cite{Mon13}.
Moreover much is known about the eigenvalue spectrum of such matrices. There is a very important spectral phase transition that exists: if $\lambda < 1$, the top eigenvector of the adjacency matrix, $\matvar{A}$, is orthogonal to the vector we wish to reconstruct (i.e. $\innerP{\vecvar{v}_{1}(\matvar{A}),\vecvar{u} } \approx 0$), whereas, if $\lambda > 1$, the top eigenvector of $\matvar{A}$ is correlated with the vector we wish to reconstruct and $\innerP{\vecvar{v}_{1}(\matvar{A}),\vecvar{u}} \approx (1 - \lambda^{-2})$ \cite{Mon13}.
In the latter regime, one eigenvector pops out of the semicircle lobe, as illustrated in \cref{fig:spectralPhaseTransitionWignerPlots}.
This particular eigenvector is associated with the eigenvalue $\lambda + \lambda^{-1}$ \cite{Mon13}.

%---   FIGURE
\begin{figure}
	\centering
	\begin{subfigure}{.5\textwidth}
		\centering
		\includegraphics[width=0.8\linewidth]{figures/spectralPhaseTransitionWigner.png}
		\caption{}
		\label{fig:spectralPhaseTransitionWigner}
	\end{subfigure}%
	\begin{subfigure}{.5\textwidth}
		\centering
		\includegraphics[width=0.8\linewidth]{figures/spectralPhaseTransitionWigner2.png}
		\caption{}
		\label{fig:spectralPhaseTransitionWigner2}
	\end{subfigure}
	\caption[Plots illustrating spectral phase transition of Wigner Matrices.]{\label{fig:spectralPhaseTransitionWignerPlots} We plot the limiting spectral density of the normalised adjacency matrix under two regimes. We illustrate the case where $\lambda < 1$, in \subref{fig:spectralPhaseTransitionWigner} and the case where $\lambda > 1$, in \subref{fig:spectralPhaseTransitionWigner2}. The blue dot in \subref{fig:spectralPhaseTransitionWigner2} represents the eigenvalue ($\lambda + \lambda^{-1}$) associated with he eigenvector that pops-out of the main semicircle lobe. Both figures obtained from \cite{Mon13}.}
\end{figure}

This result is key since it specifically describes a threshold where traditional spectral methods such as standard Principal Component Analysis (PCA) will not work (i.e. when $\lambda < 1$) and when it will produce a reconstructed vector correlated with the ground-truth (i.e. when $\lambda > 1$).
Moreover, we now have sufficient motivation to investigate methods where we can do better; more specifically we wish to study algorithms where we can essentially `beat' this spectral threshold by producing a reconstructed vector correlated with the ground-truth in the regime where $\lambda < 1$.
There is hope of achieving the improvement over standard PCA since we observe that the structure of the principal eigenvector of the matrix has two special properties.
Firslty, it is \emphT{non-negative} (since the elements are node assignments or indicator variables and are thus either zero or one) and, secondly, it is \emphT{sparse} (since we are interested in the regime where the size of the hidden community or clique is small relative to the size of the graph).
We will see how to utilise of this observation in more detail in \cref{subsec:nonLinearPowerIteration} since it forms the basis of a class of algorithms.

%--------------------------------
%   Financial Background Section   
%--------------------------------

\section{Financial Networks}
\label{sec:financialNetworksBackground}

%-------------------------------------------
%   Prices Returns Financial Assets Sub Section
%-------------------------------------------

\subsection{Prices and Returns of Financial Assets}
\label{subsec:financialAssetsBackground}

Financial assets are instruments claiming to have monetary value that can be bought and sold. Financial assets can be separated into broad classes, with examples including stocks, bonds or real estate \cite{Kuh12d,BKM13}.
The values of these assets is reflected in their price, which varies with time. Investors may wish to decide between which asset classes to invest in at any time, a process known as \emphT{asset allocation} \cite{BKM13}.
Also, within a particular asset class, an investor wishes to allocate money to specific assets, a process known as \emphT{portfolio selection} \cite{BKM13}.
For this report, we will focus on portfolio selection of stocks in our application of community detection algorithms, due to data availability constraints; however a very similar scheme may be used to tackle asset allocation also.

Investors tend not to consider the prices of assets they have invested in, but rather the \emphT{return} generated. Let us consider a financial asset whose price at time $t$ is $p(t)$. One popular measure of return is called the \emphT{rate of return} \cite{Kuh12e,BKM13} at a time $t$, denoted by $r(t)$, which is defined as
\begin{equation}
	\label{eq:rateOfReturn}
	r(t_{0}) = \frac{p(t_{1}) - p(t_{0})}{p(t_{0})}
\end{equation}
where we can interpret $t_{1}$ as the time when the investor sold the asset, and $t_{0}$ as the time when the investor bought the asset.

Critically, the rate of return is sensitive to large changes for longer time horizons \cite{Onn02}. In particular, we can consider a different measure of return, which is equivalent to a return with a constant interest rate \cite{Onn02}. We can generalise the concept interest rates with the simple example of an investor placing money (investing) in a bank account, as explained in \cite{Kuh12c,Lue98}.
The amount of money initially invested is referred to as the \emphT{principal}. We then assume money grows by a multiplicative factor, where the gain is paid into the account by the bank. This process is often called \emphT{compounding}. The time at which the interest is compounded, is called the \emphT{compounding period}.
\emphT{Compound interest} involves interest being paid on both the principal and the accumulated interest up to the present \cite{Onn02}. Typically, we are interested in the number of compounding periods in one year (i.e. the number of times the interest on our principal is compounded each year) \cite{Kuh12c}. Denote the principal by $w_{0}$, the amount in the account at time $t$ by $w_{t}$, the interest rate by $y$ and the number of compounding periods in a year by $m$. Then the amount within the account holdings after 1 year is given by
\begin{equation}
	\label{eq:compoundInterest}
	w_{1} = w_{0} (1 + (y/m))^{m}
\end{equation}

We can imagine diving a year into infinitesimally small compounding periods, and then determine the effect of this continuous compounding by taking the limit of ordinary compounding \cite{Lue98}. Notice the total number of compounding periods in a length of $t$ years is given by $mt$. Thus the effect of continuous compounding is
\begin{equation}
	\label{eq:continuousCompounding}
	w_{t} = \lim_{m {\to} \infty} w_{0}(1 + (y/m))^{mt} = w_{0}\exp{yt}
\end{equation}

If we divide \cref{eq:continuousCompounding} by the initial investment $w_{0}$, and take the natural logarithm, we get a representation for the return, $r = yt$. This indicates that taking the natural logarithm results in a constant interest rate.
We have thus arrived at another measure for return, called the \emphT{logarithmic return}, which is defined by
\begin{equation}
	\label{eq:logarithmicReturn}
	r(t_{0}) = \natlog{p(t_{1})} - \natlog{p(t_{0})}
\end{equation}
where, once more, we can interpret $t_{1}$ as the time when the investor sold the asset, and $t_{0}$ as the time when the investor bought the asset.

There are several advantages to using logarithmic returns, as explained in \cite{QuaWp}, which we will briefly summarise.
Firstly, if we assume asset prices have a \emphT{log normal} distribution, then the logarithmic returns are conveniently normally distributed \cite{QuaWp}. The reasons why assuming a log normal distribution may be appropriate for dynamic pricing of assets is beyond the scope of this report.
Secondly, for small rates of return, the logarithmic return is approximately equal to the rate of return \cite{QuaWp}. To see this, consider the approximation result from \cref{eq:logarithApproximation} combined with \cref{eq:rateOfReturn,eq:logarithmicReturn}.
\begin{equation}
	\label{eq:logarithApproximation}
	\natlog{1 + x} \approx x \text{, for } x \ll 1
\end{equation}
Thirdly, we benefit from numerical stability since the addition of small numbers is numerically stable, whilst multiplying small numbers is subject to \emphT{arithmetic underflow} \cite{QuaWp}.

However, there are disadvantages to using the logarithmic return, including the issue that the derivation is only correct if the interest rate is constant \cite{QuaWp,Onn02}.
Nevertheless, the logarithmic return is widely used in the literature (e.g. see \cite{Onn02,OCK+02,OKK03,FPM+10,FPW+11,MG13}), and hence we shall use it for the rest of the report, and the reader should note we shall use the terms `return' and `logarithmic return' interchangeably from now on. An example plot of price and logarithmic return for a stock is shown in \cref{fig:priceAndLogReturn}.

%--- FIGURE
\begin{figure}
	\centering
	\includegraphics[width=0.8\linewidth]{figures/priceAndLogReturnTimeSeries.png}
	\caption[Example plot for price and logarithmic return]{\label{fig:priceAndLogReturn} Plots of prices and logarithmic returns for Anglo American plc (AAL) between 2004 and 2014}
\end{figure}

%-------------------------------------------
%   Mean Variance Portfolio Theory Sub Section
%-------------------------------------------

\subsection{Mean-Variance Portfolio Theory}
\label{subsec:portfolioTheoryBackground}

The term \emphT{portfolio} relates to investing in a combination of different assets.
We can characterise a portfolio by \emphT{portfolio weights}, where the weight of an asset within the portfolio is given by the ratio of the value of the position in the asset divided by the total value of the portfolio.
We are particularly interested in the return of the portfolio, which is related to the mean of the returns of the individual assets that make up the portfolio and the risk of the portfolio, which is related to the variance of the returns of the individual assets.
This is the source of the term mean-variance portfolio theory.
We realise an intuitive and inherit trade-off between risk and return; if the investor wishes to realise a larger return, he must bear a higher risk.
A more detailed explanation of this relationship is given in \cite{Lue98,Kuh12e}.
Rather, we are simply interested in finding the most efficient, or \emphT{minimum-variance portfolio} for a given requested portfolio return (that is to say, the investor requests a specific expected return, and wishes to form a portfolio that has the lowest variance of all possible portfolio that can deliver the specified expected return).
We can formalise the mean-variance portfolio setting in the following way, which summarises the explanations from \cite{Onn02,Lue98,Kuh12e}.

Let $P$ be a portfolio comprising of $n$ assets, where the return of the portfolio is denoted by $r_{P}$ and the variance of the portfolio is denoted by $\sigma^{2}_{P}$.
Denote the return of asset $i$ by $r_{i}$, the variance of the asset $i$ return by $\sigma^{2}_{i} \equiv \variance{r_{i}}$, and the weight of asset $i$ in the portfolio by $w_{i}$.
Also let $X_{0}$ denote the total amount invested in the portfolio (i.e. initial investment in the portfolio by the investor) and $X_{0i}$ represent the the amount invested in asset $i$.
We select the amounts in the assets forming the portfolio such that
\begin{equation}
	\label{eq:portfolioAmounts}
	\sum_{i=1}^{n} X_{0i} = X_{0}
\end{equation}
We define the portfolio weights using
\begin{equation}
	\label{eq:portfolioWeights}
	w_{i} = \frac{X_{0i}}{X_{0}} \text{ for } i = 1,\dots,n
\end{equation}
meaning that
\begin{equation}
	\label{eq:portfolioWeightsSum}
	\sum_{i=1}^{n} w_{i} = 1
\end{equation}
Notice that a negative weight indicates a \emphT{short position} in that asset, and that the returns of the individual assets and portfolio are random variables.

In particular we can represent the return of the portfolio by
\begin{equation}
	\label{eq:portfolioReturn}
	r_{P} = \sum_{i=1}^{n} w_{i} r_{i}
\end{equation}
By using \cref{eq:portfolioReturn}, we obtain the expected return of the portfolio
\begin{equation}
	\label{eq:portfolioExpectedReturn}
	\expectation{r_{P}} = \sum_{i=1}^{n} w_{i} \expectation{r_{i}}
\end{equation}
and the variance of the portfolio return
\begin{equation}
	\label{eq:portfolioVariance}
	\variance{r_{P}} \equiv \sigma^{2}_{P}  = \sum_{i=1}^{n} \sum_{j=1}^{n} w_{i} w_{j} \rho_{ij} \sigma_{i} \sigma_{j}
\end{equation}
where $\rho_{ij}$ is the correlation coefficient of the returns of assets $i$ and $j$, defined as
\begin{equation}
	\label{eq:correlationCoefficient}
	\rho_{ij} = \frac{\covariance{r_{i},r_{j}}}{\sigma_{i} \sigma_{j}}
\end{equation}
Note that the standard deviation of the returns of the portfolio (or any asset) is often called its \emphT{volatility}.

This formulation serves a key question; given estimates of each assets returns, variances and covariances (which one can obtain from historical data), how does one pick a selection of these assets, for any given time period, in order to form the best portfolio for the investor?
We see in \cref{eq:portfolioVariance}, that by simply investing in assets which have a lower correlation with one another (i.e. a lower value of $\rho_{ij}$), we can reduce the variance, and thus, the risk of the portfolio.
This process is known as \emphT{diversification} \cite{Lue98,BKM13}.
Therefore, for any given time period (and possibly dynamically), finding groups of assets, where the returns have higher correlation within groups and lower correlation between groups would help by presenting `baskets' of assets that the investor can pick from knowing selecting from a range of baskets would be beneficial (of course which assets to select from inside the basket relates to the risk-return trade off).
This serves as the main motivation for the application of community detection algorithms within financial networks.

%-------------------------------------------
%   Constructing Financial Networks Sub Section
%-------------------------------------------

\subsection{Constructing Financial Networks}
\label{subsec:financialNetworksConstructionBackground}

From \cref{subsec:portfolioTheoryBackground}, we understand one way to help minimise risk in constructing portfolios involves analysing the correlation coefficients of returns between two assets.
In order to study all possible correlations between all available assets, we construct a weighted, undirected and fully-connected network of assets, which we call the \emphT{financial network}. The following model has been considered by \cite{PGR+99,OCK+02,OKK03,FPM+10,MG13}.

Let us consider the situation where the investor is faced with $n$ financial assets, and has access to historical price data for all these assets for $T$ time steps.
The time steps may be trading days, or weeks, for instance, and the appropriate choice will depend on the type of assets available.

We proceed to construct a graph with $n$ nodes, where each nodes represents an asset, and assign to the $i$-th node a single time series, denoted by $X_{i}$, which is defined as
\begin{equation}
	\label{eq:singleTimeSeries}
	X_{i} = \{x_{i}(1),\dots,x_{i}(T)\}
\end{equation}
where $x_{i}(t)$ describes the logarithmic return of asset $i$ at time $t$, defined by \cref{eq:logarithmicReturn}.
This time series describes the evolution of the logarithmic return of the asset over $T$ time steps.
We then model the weight of an edge connecting nodes $i$ and $j$ of the graph by the cross-correlation between the time series corresponding to assets $i$ and $j$.
We form a cross-correlation matrix, denoted by $\matvar{C}$, whose elements are defined by
\begin{equation}
	\label{eq:crossCorrelationMatrix}
	C_{ij} = \frac{\mean{X_{i} X_{j}} - \mean{X_{i}} \mean{X_{j}}}{\sqrt{\left[ \mean{X_{i}^{2}} - \mean{X_{i}}^{2} \right] \left[ \mean{X_{j}^{2}} - \mean{X_{j}}^{2} \right]}}
\end{equation}
where the $\mean{\cdots}$ notation denotes a time average, so that
\begin{equation}
	\label{eq:temporalMean}
	\mean{X_{i}} = \frac{1}{T} \sum_{t=1}^{T} x_{i}(t)
\end{equation}
\begin{equation}
	\label{eq:temporalMeanSquare}
	\mean{X_{i}^{2}} = \frac{1}{T} \sum_{t=1}^{T} x_{i}^{2}(t)
\end{equation}
\begin{equation}
	\label{eq:temporalMeanProduct}
	\mean{X_{i}X_{j}} = \frac{1}{T} \sum_{t=1}^{T} x_{i}(t)x_{j}(t)
\end{equation}
We also assume each time series $X_{i}$ has been standardised (before we assign to node $i$) by using
\begin{equation}
	\label{eq:standardiseTimeSeries}
	X_{i} \coloneqq \frac{X_{i} - \mean{X_{i}}}{\sqrt{\mean{X_{i}^{2}} - \mean{X_{i}}^{2}}}
\end{equation}
so that
\begin{equation}
	\label{eq:zeroTemporalMean}
	\mean{X_{i}} = 0
\end{equation}
\begin{equation}
	\label{eq:unitTemporalVariance}
	\mean{X_{i}^{2}} - \mean{X_{i}}^{2} = 1
\end{equation}
Note that the cross correlation values is just a sample estimate for the correlation coefficient, $\rho_{ij}$, used in \cref{subsec:financialNetworksConstructionBackground}, calculated from the historical data.

We can then characterise the financial network by the correlation matrix, which we also refer to as the network's weighted adjacency matrix.

In \cref{fig:exampleCrossCorrelationMatrix}, we have plotted a correlation matrix using data of 80 stocks listed on the FTSE 100 (see \cref{app:listFTSE100Stocks} for a list) between 2011 and 2013. Notice that the main diagonal has all elements equal to one, as you would expect, and that there are very few negative elements (i.e. very few assets that are anti-correlated with one another).

%--- FIGURE
\begin{figure}
	\centering
	\includegraphics[width=0.7\linewidth]{figures/correlationMatrix_FTSE100_n_80_T_2501.png}
	\caption[Example plot for a correlation matrix]{\label{fig:exampleCrossCorrelationMatrix} Example of a correlation matrix. Evaluated from an ensemble of 80 stocks listed on the FTSE 100 (see \cref{app:listFTSE100Stocks} for a list) using data between 01/01/2011 and 01/01/2013.}
\end{figure}

The problem statement can now summarised.
Given a financial network, how can we group nodes into communities where correlations are higher within the communities and lower between the communities?
Contrary to graphs with community structure described in \cref{sec:communityStructureBackground}, the weights of the edges rather than the topology of the network are crucial in determining community memberships.
In other words, we focus solely on the weighted adjacency matrix of the graph.
Also the reader should note that the correlations between asset returns will vary over time, and thus representing this relationship dynamically (rather than over a one long period of time) is very important since investors may wish to change their positions in order to react to the dynamics of market conditions.


% Community Detection Algorithms
% Chapter 3

\chapter{Community Detection Algorithms}

\label{cha:communityDetectionAlgorithms}

%----------------------------------------------------------------------------------------

In this chapter we introduce several community detection algorithms present in the literature that can be used to detect communities based upon different approaches.
We introduce spectral clustering, modularity-based optimisation, non-linear power iteration and message-passing algorithms, with specific reference to their application on generative block models.
We seek to explain sufficient intuition motivating the algorithms in addition to summarising their mathematical derivations.

%-----------------------------------------------------
%   Spectral Clustering Section
%-----------------------------------------------------

\section{Spectral Clustering}
\label{sec:spectralClustering}

The basis of all spectral clustering algorithms is the transformation of a set of variables into the set of points in space whose coordinates are elements of eigenvectors of a matrix, and then the clustering of these points using well-known clustering algorithms \cite{Lux06,For10}.

We consider the spectral clustering algorithm described in \cite{Lux06,For10}, whose intuition is explained in \cite{Lux06,Spi07,For10}.

Firstly, we compute the Laplacian matrix of the network, using \cref{def:unnormalisedLaplacianMatrix}, where we assume $n$ nodes in the graph, as usual.
We then compute the $k$ eigenvectors of the Laplacian matrix associated with the $k$ largest eigenvalues.
Denote the eigenvectors by $\vecvar{u}_{1},\dots,\vecvar{u}_{k}$ and the eigenvalues by $\lambda_{1},\dots,\lambda_{k}$.
The eigenvectors represent a $k$-tuple of real numbers associated with each vertex in the graph. 
We think of this association as a mapping from the vertices into a $k$-dimensional space.
This embedding is characterised by $F : V \rightarrow \realsR^{k}$ where $F(i) = (\vecvar{u}_{1}^{(i)},\dots,\vecvar{u}_{k}^{(i)})$ and $\vecvar{u}_{j}^{(i)}$ denotes the $i$-th element of the $j$-th eigenvector.
We have essentially represented node $i$ of the graph as a point in a $k$-dimensional space where the coordinates are the $i$-th elements of all the top $k$ eigenvectors of the Laplacian matrix.
Finally, we apply the embedding as input to the popular \emphT{k-means clustering} algorithm.
The cluster memberships of the $n$ data points are precisely the estimated node assignments for the initial network.
Notice, from the definition of the Laplacian matrix, that the vector of all ones (i.e. a vector with every element equal to one) is the principal eigenvector, so these embedded values remain the same for all nodes.
Therefore, knowledge of this eigenvector does not help discriminate between different vertices, and hence the information is not useful.
Thus we shall only apply the embedding to the top $k-1$ eigenvectors (i.e. the top $k$ eigenvectors excluding the all-ones vector), and can therefore represent the mapping in a $k-1$ dimensional space.

To visualise what such an embedding looks like, refer to \cref{fig:SpectralClusteringEmbeddingVisualisation}, where we generated an example graph using the planted partition model and 3 communities (this is purposely chosen so we can easily identify the embedding in a 2-dimensional space).
The coordinates for each point are the corresponding entries in the 2 eigenvectors of the Laplacian matrix considered.
We labelled the ground-truth node assignments by colour in the figure (i.e. points with the same colour represent nodes belonging to the same community in the underlying graph), and we can see that a k-means clustering algorithm would be applied to return cluster memberships that match the ground-truth, since the data points corresponding to different clusters are well separated.

%---   FIGURE
\begin{figure}
	\centering
	\includegraphics[width=0.7\linewidth]{figures/embeddedVectorsModularityMethod_pin_0_8_pout_0_2.png}
	\caption[Visualisation of spectral clustering embedding.]{\label{fig:SpectralClusteringEmbeddingVisualisation} A visualisation of the embedding for the spectral clustering algorithm. The graph was generated using the planted partition model with $n=150$, $k=3$, $p_{in}=0.8$ and $p_{out}=0.2$. We have chosen $3$ distinct communities since we can easily display the embedding in a 2-dimensional space. The coordinates for each point are the corresponding entries in the 2 eigenvectors of the Laplacian matrix considered. We label the ground-truth node assignments by colour (i.e. points with the same colour represent nodes belonging to the same community in the graph), and we can see that a k-means clustering algorithm would be applied to return cluster memberships that match the true community memberships.}
\end{figure}

Note that the algorithm does require to compute $k$ eigenvectors of a matrix, and this can be achieved using the power method.
Also, the accuracy of the algorithm will largely depend on the k-means algorithm which has been shown to converge to local minima in a cost measure (rather than global), but despite this, has been shown to work well in practical applications \cite{Lux06,For10}.

%-----------------------------------------------------
%   Modularity-based Optimisation Section
%-----------------------------------------------------

\section{Modularity-based Optimisation}
\label{sec:modularityBasedOptimisation}
 
We introduce the algorithms, firstly considered by \cite{New06a,New06b}, by explaining the intuition behind `good' community partitions.
In essence, the key ingredient involves determining partitions of the network where there are fewer edges \emphT{than expected} between nodes belonging to different communities.
For instance, if the number of links between nodes associated between different communities is approximately the same as what one would expect to find given random assignment of edges within the network, then it is unlikely this provides evidence of meaningful community structure \cite{New06b}.
Moreover, we can also consider partitions where there are more edges than expected between nodes belonging to the same community.
\begin{definition}
	\label{def:nullModel}
	The \emphT{null model} with respect to a network, whose adjacency matrix is given by  $\matvar{A}$, is the random graph denoted by $\graphvar{G}$, where each edge has a probability of $\frac{d_{i}d_{j}}{2m}$ of occurring. $d_{i}$ is the degree of node $i$ and $2m \equiv \sum_{ij} A_{ij}$.
\end{definition}
The null model defined above is proposed as a baseline distribution if edges were randomly placed within the network.
\begin{definition}
	\label{def:modularity}
	Given a partition, $\vecvar{\sigma}$, of a network, the \emphT{modularity} is defined as: $Q(\vecvar{\sigma}) = \frac{1}{2m} \sum_{ij} \left(A_{ij} - \frac{d_{i}d_{j}}{2m} \right) \delta(\sigma_{i},\sigma_{j})$.
\end{definition}
The modularity is therefore considered a cost function for a partition of the network where larger modularity values indicate stronger community structure \cite{New06a}.
\begin{definition}
	\label{def:modularityMatrix}
	The \emphT{modularity matrix} is denoted by $\matvar{B}$, whose elements $B_{ij}$ are defined by $B_{ij} = A_{ij} - \frac{d_{i}d_{j}}{2m}$.
\end{definition}
The aim of modularity optimisation algorithms is to find a partition of the network with the maximum value of modularity associated.
Since searching over all possible partitions is exponential in the number of nodes of the network, the problem is NP-hard computationally \cite{New06b}.
Thus, we seek approximate methods that provide near-optimal solutions.

In the literature, there exists a variety of approximation algorithms for accurate and fast modularity optimisation, such as greedy algorithms, simulated annealing, spectral algorithms and extremal optimisation \cite{For10}.
Within this report, we describe all these algorithms but only implement and test the greedy agglomerative method on synthetic data, since we consider it to be a faster version whilst maintaining similar accuracy to other modularity optimisation algorithms.

%-------------------------------------------
%   Greedy Algorithms Sub Section
%-------------------------------------------

\subsection{Greedy Algorithm}
\label{subsec:greedyAlgorithm}

The greedy algorithm of Clauset et al. \cite{CNM04} starts with all nodes as single groups and successively mergers two groups to form a larger community such that the modularity of the new partition increases after the joining \cite{CNM04,For10}.
Moreover, the algorithm keeps, permanently, the merger with the largest increase in modularity (hence at each step we compute $\Delta Q$, the change in modularity, using \cref{def:modularity}).
This is iterated until no further increase in modularity is possible \cite{CNM04}.
Note that for a network with $n$ nodes and $m$ edges, the algorithm has complexity $\bigO{(m+n)n}$, or $\bigO{n^{2}}$ for a sparse graph \cite{For10}.

A different greedy algorithm has been proposed by Blondel et al. \cite{BGL+08}, that is also applicable for weighted networks \cite{For10}.
This algorithm is commonly known as the \emphT{Louvain method}, and we shall also often refer to the algorithm by this name.
We initialise each node to belong to an individual community, and then repeat the following two phases until there is no further increase in modularity possible.
In the first phase, we sequentially consider each node, and given node $i$, we compute the increase in modularity, $\Delta Q$, that results from moving node $i$ into a neighbour community, and then permanently select the transition that yields the greatest increase in modularity \cite{BGL+08,For10}.
In the second phase, two communities are connected if an edge exists between any node belonging to the communities \cite{BGL+08,For10}.
\Cref{fig:LouvainMethodIllustration} illustrates the two phases of the algorithm on an example network. This figure is taken from the Blondel et al. reference \cite{BGL+08}.

%---   FIGURE
\begin{figure}
	\centering
	\includegraphics[width=0.9\linewidth]{figures/louvainMethodIllustration.png}
	\caption[Illustration of greedy algorithm for modularity optimisation.]{\label{fig:LouvainMethodIllustration} Illustration of the two phases of the greedy method of \cite{BGL+08}. The first phase involves optimisation of modularity using local changes, and the second phase aggregates the nodes into communities. The two phases are repeated until no further improvement of modularity is possible. For this example, only two passes are required until termination. This figure is reprinted from the Blondel et al., reference \cite{BGL+08}.}
\end{figure}

%-------------------------------------------
%   Simulated Annealing Sub Section
%-------------------------------------------

\subsection{Simulated Annealing}
\label{subsec:simulatedAnnealing}

The simulated annealing algorithm of Kirkpatrick et al. \cite{KGV83} is an iterative procedure that explores a space of possible states looking for the global optimum of modularity, which we denote by $Q$ \cite{KGV83,For10}.
Updates from one state to another are accepted with probability 1 if the transition results in an increase in the modularity.
Otherwise, the update is only accepted with a small probability $\exp{-\beta \Delta Q}$, where $\Delta Q$ is the change in modularity (i.e. value of modularity after the transition minus the value before) and $\beta$ represents the inverse-temperature of the system \cite{KGV83,For10}.
The idea behind accepting a transition that results in a decrease in modularity with a small probability is to increase the chance of finding the global maximum (i.e. decrease the chance of converging towards local maxima) \cite{For10}.
The algorithm converges to a stable state at some point, depending on the number of states explored and how $\beta$ is varied, but it can be a good approximation for $Q$.

A more recent implementation by Guimera et al. \cite{GSA04} consists of iterations that involve both individual and collective steps.
Within the individual step, an individual node is moved to a community at random, whilst the collective step consists of merging two communities or splitting one community \cite{GSA04}.
Typically, each iteration involves $n^{2}$ individual steps and $n$ collective steps, where $n$ represents the number of nodes in the network \cite{For10}.
This method can approximate the true maximum of modularity very accurately, and note that, due to the variation of parameter selection (such as initial temperature and inverse-temperature chosen), an exact complexity cannot be estimated, but it is typically very slow and hence should only be used on small graphs \cite{For10}.

%-------------------------------------------
%   Extremal Optimisation Sub Section
%-------------------------------------------

\subsection{Extremal Optimisation}
\label{subsec:extremalOptimisation}

The extremal optimisation algorithm of Duch and Arenas \cite{DA05} is a heuristic search method that involves recursively bi-partitioning the network \cite{DA05,For10}.
It begins with a random partition and uses the contribution of each node to the modularity as a fitness measure with the movement of nodes with the lowest fitness value.
The fitness function value of a node $i$ is given by
\begin{equation}
	\label{eq:fitnessFunction}
	q_{i} = \kappa_{\sigma(i)} - d_{i}e_{\sigma(i)}
\end{equation}
where $d_{i}$ is the degree of node $i$, $\kappa_{\sigma(i)}$ is the number of neighbours node $i$ has in the community it belongs to, and $e_{\sigma(i)}$ is the fraction of edges in the network that connects at least one node which belongs to the community of node $i$ \cite{DA05,For10}.
Using this notation, one can re-write the modularity by $Q = \frac{1}{2m} \sum_{i} q_{i}$.
We also normalise the variables $q_{i}$ by dividing by $d_{i}$ to obtain
\begin{equation}
	\label{eq:normalisedFitnessFunction}
	\rho_{i} = \frac{\kappa_{\sigma(i)}}{d_{i}} - e_{\sigma(i)}
\end{equation}
so that $-1 \leq \rho_{i}  \leq 1$ for all $i$.
Therefore we have expressed the global cost function in terms of a sum over all vertices (through the local variables $\rho_{i}$) and, hence, we can optimise the global variable, $Q$, by optimising over the local variables \cite{For10}.
At each iteration of the algorithm we calculate $\rho_{i}$ for every node $i$ and move the node with the lowest value to the other community.
Note that this transition alters the overall partition so the fitness values need to be re-evaluated, whilst we repeat this process until no further improvement in the modularity is possible \cite{DA05,For10}.
Extremal optimisation has empirically shown to achieve similar accuracy to simulated annealing but with a faster run time \cite{For10}.

%-------------------------------------------
%   Spectral Algorithm Sub Section
%-------------------------------------------

\subsection{Spectral Algorithm}
\label{subsec:spectralAlgorithm}

We shall describe the spectral method of Newman \cite{New06a} using the derivation explained in \cite{New06a,New06b,For10}, and by firstly considering networks with only two ground-truth communities.
Recall the notation used for modularity and the modularity matrix in \cref{def:modularity,def:modularityMatrix}, and let $\vecvar{\sigma}$ represent the vector of node assignments, where $\sigma_{i} = 1$ if node $i$ belongs to class 1 and $\sigma_{i} = -1$ if it belongs to class 2.
Then the modularity can be written as
\begin{equation}
	\label{eq:rewriteModularity}
	\begin{split}
		&Q = \frac{1}{2m} \sum_{i,j} \left(A_{ij} - \frac{d_{i}d_{j}}{2m} \right) \delta(\sigma_{i},\sigma_{j}) \\
		&= \frac{1}{4m} \sum_{i,j} \left(A_{ij} - \frac{d_{i}d_{j}}{2m} \right) (\sigma_{i}\sigma_{j} + 1) \\
		&= \frac{1}{4m} \sum_{i,j} B_{ij} \sigma_{i} \sigma_{j} \\
		&= \frac{1}{4m} \transpose{\vecvar{\sigma}} \matvar{B} \vecvar{\sigma}
	\end{split}
\end{equation}
We can rewrite $\vecvar{\sigma}$ as a linear combination of the eigenvectors of $\matvar{B}$, which we denote by $\vecvar{u}_{1},\dots,\vecvar{u}_{n}$ (where we label in decreasing order corresponding to the magnitude of its associated eigenvalue), so that
\begin{equation}
	\label{eq:rewriteNodeAssignments}
	\vecvar{\sigma} = \sum_{i=1}^{n} (\transpose{\vecvar{u}_{i}} \vecvar{\sigma}) \vecvar{u}_{i}
\end{equation}
Using this result in \cref{eq:rewriteModularity} yields
\begin{equation}
	\label{eq:rewriteModularity2}
	Q = \sum_{i} (\transpose{\vecvar{u}_{i}} \vecvar{\sigma}) \transpose{\vecvar{u}_{i}} \matvar{B} \sum_{j} (\transpose{\vecvar{u}_{j}} \vecvar{\sigma}) \vecvar{u}_{j} = \sum_{i=1}^{n} (\transpose{\vecvar{u}_{i}} \vecvar{\sigma})^{2} \lambda_{i}
\end{equation}
where $\lambda_{i}$ is the eigenvalue of $\matvar{B}$ associated with the eigenvector $\vecvar{u}_{i}$.
We aim to maximise the modularity by choosing the elements of $\vecvar{\sigma}$. From \cref{eq:rewriteModularity2}, we see this can be achieved by increasing the weights of the largest (i.e. most positive) eigenvalues.
However, we cannot just set $\vecvar{\sigma}$ to be proportional to the largest eigenvector, $\vecvar{u}_{1}$, as we imposed each element to be either +1 or -1.
Instead, we seek an approximate method, where we proceed to set the values of $\sigma_{i}$ based upon the sign of the $i$-th component of $\vecvar{u}_{1}$.
Essentially, the algorithm involves computing the leading eigenvector of the modularity matrix and then partitioning the nodes of the network into two groups according to the signs of the corresponding elements in the eigenvector.

In order to extend this approach for networks which contain more than two communities, we repeat this procedure for dividing any one community into two communities until no further sub-division increases the value of modularity, at which point the algorithm terminates.

The spectral method for optimising modularity is quite fast, since computing the leading eigenvector of the modularity matrix can be computed using the well-known power method.
Due to the special structure of the modularity matrix, the computation of the leading eigenvector takes $\bigO{m+n}$ time, so that one partition of the network takes $\bigO{n(m+n)}$ time or $\bigO{n^{2}}$ for a sparse graph \cite{For10}.
As we need to repeatedly partition the network in order to optimise the modularity, the overall complexity is $\bigO{dn(m+n)}$ where $d$ represents the depth of the hierarchical division.
Typically, in practice $d \approx \natlog{n}$, so that for sparse graphs, the total time taken for this spectral algorithm is approximately $\bigO{n^{2}\natlog{n}}$.
The spectral method is faster than simulated annealing and extremal optimisation, although not as fast as the greedy algorithm, based on empirical results \cite{For10}.
An added benefit is the extensibility of the spectral method to applications with weighted networks.

%-----------------------------------------------------
%   Belief Propagation Algorithm Section
%-----------------------------------------------------

\section{Belief Propagation Algorithm}
\label{sec:beliefPropagationAlgorithm}

The belief propagation (BP) algorithm of \cite{Has06,DKM+13} is designed to infer the group assignment from an instance of a graph generated by the planted partition model.
We shall describe Decelle et al.'s BP algorithm \cite{DKM+13} for the particular case of the planted partition model, but note that message passing algorithms such as BP are very similar in nature and incredibly useful in many other applications outside the problem of community detection \cite{For10}.
We also notify the reader that this algorithm is derived from principles in statistical physics, that we do not wish to delve into too much detail (since it is mostly inconsequential to the motivation of this report), and therefore shall present a higher-level overview and less detailed summary of the derivation by Decelle et al. \cite{DKM+13}. 

Recall the notation used for introducing the planted partition model in \cref{subsec:plantedPartitionModel}.
We begin by realising the probability the planted partition model with parameters $\theta = \{k,n_{a},\{P_{ab}\}\}$ generates a graph $\graphvar{G}$ consisting of $n$ nodes, with an associated adjacency matrix $\matvar{A}$ and estimated node assignments ${q_{i}}$ is given by
\begin{equation}
	\label{eq:probabilityPPMGeneratingGraph}
	\probability{\graphvar{G},\{q_{i}\} \given \theta} = \prod_{i \neq j} \left[ P_{q_{i}q_{j}}^{A_{ij}} (1-P_{q_{i}q_{j}})^{1-A_{ij}} \right] \prod_{i} n_{q_{i}}
\end{equation}
Assuming we know the underlying parameters of the block model, $\theta$, as well as observing the graph $\graphvar{G}$, we form the probability distribution over the group assignments by applying Bayes' theorem
\begin{equation}
	\label{eq:probabilityDistributionGroupAssignments}
	\probability{\{q_{i}\} \given \graphvar{G},\theta} = \frac{\probability{\graphvar{G},\{q_{i}\} \given \theta}}{\sum_{t_{i}} \probability{\graphvar{G},\{q_{i}\} \given \theta}}
\end{equation}
where $\{ t_{i} \}$ now represent the ground-truth node assignments (instead of $\{ \sigma_{i} \}$) for convenience.

We make an important connection between this problem and a result studied in statistical physics.
The \emphT{Boltzmann distribution} of a generalised Potts model (consider $\{ \sigma \}$ simply as a set of discrete variables) with Hamiltonian $H(\{ \sigma \})$ at inverse temperature $\beta$ is given by
\begin{equation}
	\label{eq:boltzmannDistributionPottsModel}
	\mu(\{ \sigma \}) = \frac{\exp{- \beta H(\{ \sigma \})}}{\sum_{\{ \sigma \}} \exp{- \beta H(\{ \sigma \})}}
\end{equation}
where each assignment of the variable $\{ \sigma \}$ has a weight $\exp{- \beta H(\{ \sigma \})}$ known as the \emphT{Boltzmann weight} and $Z(\beta) = \sum_{\{ \sigma \}} \exp{- \beta H(\{ \sigma \})}$ is called the \emphT{partition function} (i.e. the sum of the Boltzmann weights over all possible configurations) \cite{Pru14a,Pru14b,Sus}.
We now use that \cref{eq:probabilityDistributionGroupAssignments} corresponds to a Boltzmann distribution of a generalised Potts model at unit temperature (i.e. $\beta = 1$) with Hamiltonian given by
\begin{equation}
	\label{eq:hamiltonianDistributionGroupAssignments}
	H(\{ q_{i} \} \given \graphvar{G}, \theta) = - \sum_{i} \natlog{n_{q_{i}}} - \sum_{i \neq j} \left[ A_{ij} \natlog{c_{q_{i}q_{j}}} + (1-A_{ij}) \natlog{1 - \frac{c_{q_{i}q_{j}}}{n}} \right]
\end{equation}
where we define $c_{ab} = n P_{ab}$ to be convenient notation for sparse networks.
The corresponding Boltzmann distribution is
\begin{equation}
	\label{eq:boltzmannDistributionGroupAssignments}
	\mu(\{q_{i}\} \given \graphvar{G}, \theta) = \probability{\{q_{i}\} \given \graphvar{G},\theta}  = \frac{\exp{- H(\{q_{i}\} \given \graphvar{G}, \theta)}}{\sum_{\{q_{i}\}} \exp{- H(\{q_{i}\} \given \graphvar{G}, \theta)}}
\end{equation}
and the corresponding partition function is
\begin{equation}
	\label{eq:partitionFunctionGroupAssignments}
	Z(\graphvar{G}, \theta) = \sum_{\{q_{i}\}} \exp{- H(\{q_{i}\} \given \graphvar{G}, \theta)}
\end{equation}

The BP algorithm is essentially an iterative produce used to compute the partition function by ignoring the correlation between neighbours of a node while conditioning on its label \cite{DKM+13}.
Moreover, \cite{DKM+13} states that these correlations do not exist if the network of interactions between nodes is a tree and that, if the network observed is locally treelike, then the correlation terms become negligible making the BP algorithm exact in the asymptotic limit (i.e. as $n \rightarrow \infty$).
We can derive the BP algorithm update equations for the case of an undirected network (the directed case is a little more complex with additional equations).
We write the BP algorithm as a set of \emphT{messages}, denoted by $m_{q_{i}}^{i \rightarrow j}$, that are essentially marginal probabilities of a node $i$ belonging to a community $q_{i}$ excluding the evidence from its neighbour node $j$ \cite{DKM+13}.
The messages that govern this algorithm (similar to the derivation for loopy belief propagation \cite{Gil14a}) are then given by
\begin{equation}
	\label{eq:beliefPropagationMessageEquation}
	m_{u_{i}}^{i \rightarrow j} = \frac{1}{Z^{i \rightarrow j}} n_{u_{i}} \prod_{k \in N(i) \setminus j} \left[ \sum_{u_{k}} c_{u_{i},u_{k}}^{A_{ik}} \left(1 - \frac{c_{u_{i},u_{k}}}{n} \right)^{1-A_{ik}} m_{u_{k}}^{k \rightarrow i} \right]
\end{equation}
where $N(i)$ denotes the neighbourhood of node $i$, $\{ u_{i} \}$ represents the estimated node assignments at a particular iteration step of the algorithm and $Z^{i \rightarrow j}$ is a constant that normalises the messages ensuring they form a probability distribution (i.e. so that $\sum_{u_{i}} m_{u_{i}}^{i \rightarrow j} = 1$) \cite{DKM+13}.
In essence, this is the normalised belief in the node $i$ belonging to a community $u_{i}$ excluding the evidence from the node $j$ \cite{Gil14a}.
The algorithm involves applying \cref{eq:beliefPropagationMessageEquation} iteratively until a fixed point is reached with messages $\{ m_{q_{i}}^{i \rightarrow j}  \}$ \cite{DKM+13}.
We can write the marginal probability (or normalised belief) of node $i$ belonging to community $u_{i}$, denoted by $b_{u_{i}}^{i}$, as
\begin{equation}
	\label{eq:beliefPropagationBeliefEquation}
	b_{u_{i}}^{i} = \frac{1}{Z^{i}} n_{u_{i}} \prod_{k \in N(i)} \left[ \sum_{u_{k}} c_{u_{i},u_{k}}^{A_{ik}} \left(1 - \frac{c_{u_{i},u_{k}}}{n} \right)^{1-A_{ik}} m_{u_{k}}^{k \rightarrow i} \right]
\end{equation}
where, once more, we use a normalisation constant $Z^{i}$ \cite{DKM+13}.

The update for each step iteratively has $\bigO{n^{2}}$ complexity, since there are messages between every pair of nodes, but, for large sparse networks (i.e. for large $n$ and $c_{ab} = \bigO{1}$), we ignore terms of order in $n$ \cite{DKM+13}.
This implies a node simply sends the same message to all non-adjacent nodes (which is described as an external field in statistical physics) and thus the algorithm only needs to consider a number of messages equal to twice the number of edges in the networks, hence each iteration requires $\bigO{n}$ time \cite{DKM+13}.
We can understand this by considering the messages for two cases. Firstly, if nodes $i$ and $j$ are not adjacent,
\begin{equation}
	\label{eq:beliefPropagationMessageEquation2}
	m_{u_{i}}^{i \rightarrow j} = \frac{1}{Z^{i \rightarrow j}} n_{u_{i}} \prod_{k \not\in N(i) \setminus j} \left[ 1 - \frac{1}{n} \sum_{u_{k}} c_{u_{k}u_{i}} m_{u_{k}}^{k \rightarrow i} \right] \prod_{k \in N(i)} \left[ \sum_{u_{k}} c_{u_{k}u_{i}} m_{u_{k}}^{k \rightarrow i} \right] = b_{u_{i}}^{i} + \bigO{\frac{1}{n}}
\end{equation} 
 where the messages are independent of $j$ to the leading order \cite{DKM+13}. Now, if nodes $i$ and $j$ are adjacent in the network
\begin{equation}
	\label{eq:beliefPropagationMessageEquation3}
	m_{u_{i}}^{i \rightarrow j} = \frac{1}{Z^{i \rightarrow j}} n_{u_{i}} \prod_{k \not\in N(i)} \left[ 1 - \frac{1}{n} \sum_{u_{k}} c_{u_{k}u_{i}} m_{u_{k}}^{k \rightarrow i} \right] \prod_{k \in N(i) \setminus j} \left[ \sum_{u_{k}} c_{u_{k}u_{i}} m_{u_{k}}^{k \rightarrow i} \right] 
\end{equation}
We can now re-write \cref{eq:beliefPropagationBeliefEquation}  as
\begin{equation}
	\label{eq:beliefPropagationMessageEquation4}
	m_{u_{i}}^{i \rightarrow j} = \frac{1}{Z^{i \rightarrow j}} n_{u_{i}} \exp{-h_{u_{i}}} \prod_{k \in N(i) \setminus j} \left[ \sum_{u_{k}} c_{u_{k}u_{i}} m_{u_{k}}^{k \rightarrow i} \right] 
\end{equation}
where we ignored the $\bigO{\frac{1}{n}}$ term and have defined the external field by
\begin{equation}
	\label{eq:beliefPropagationExternalField}
	h_{u_{i}} = \frac{1}{n} \sum_{k} \sum_{u_{k}} c_{u_{k}u_{i}} b_{u_{k}}^{k}
\end{equation}
as explained by \cite{DKM+13}. The marginal probabilities can now also be re-written as
\begin{equation}
	\label{eq:beliefPropagationBeliefEquation2}
	b_{u_{i}}^{i} = \frac{1}{Z^{i}} n_{u_{i}} \prod_{k \in N(i)} \exp{-h_{u_{i}}} \prod_{j \in N(i)} \left[ \sum_{u_{j}} c_{u_{j}u_{i}} m_{u_{j}}^{j \rightarrow i} \right] 
\end{equation}
\Cref{eq:beliefPropagationMessageEquation4,eq:beliefPropagationExternalField,eq:beliefPropagationBeliefEquation2} define the crucial steps in each iteration of the BP algorithm, which is called `\textsc{BP-INFERENCE}' in \cite{DKM+13}.
The precise details can be seen in \cite{DKM+13}, but we summarise the key steps.
We start with the messages as a random vector, compute the initial marginal probabilities and external field.
We then iteratively, until a convergence criterion is met and/or for a fixed number of steps, apply the three update equations.
Finally, we output the estimated group assignment by using $\{ q_{i} \} = \argmax_{q} b_{q}^{i}$ (i.e. make the assignment of node $i$ to the group that maximises the marginal probabilities of node $i$ belonging to any of the groups).
As \cite{DKM+13} have analysed, the main body of the algorithm has $\bigO{n}$ complexity, while the number of iterations required until convergence is not known exactly and varies for different networks, so choosing an appropriate value for the maximum number of iterations based on a training set of networks would be practical.

Recall that, for the above derivations, we assumed knowledge of the underlying parameters of the block model, $\theta$.
However, in practice, given a new network, this is not the case, so we need to be able to learn the underlying parameters of the network before applying the BP inference algorithm.
Decelle et al. \cite{DKM+13} do provide another BP algorithm for learning or estimating the parameters that is based on the popular \emphT{expectation-maximisation} (EM) algorithm.
We do not aim to derive the update equations for this case, though, but will rather explain the intuition behind it.
In order to infer the parameters, we can aim to find the maximum a-posteriori estimator.
Let $\widehat{\theta}$ denote the estimator for the parameters, then $\widehat{\theta} = \argmax_{\theta} \probability{\theta \given \graphvar{G}}$.
Applying Bayes' theorem we find that
\begin{equation}
	\label{eq:inferenceParametersBayesTheorem}
	\probability{\theta \given \graphvar{G}} = \frac{\probability{\graphvar{G} \given \theta} \probability{\theta}}{\probability{\graphvar{G}}} = \frac{\probability{\theta}}{\probability{\graphvar{G}}} \sum_{\{ q_{i} \}} \probability{\graphvar{G},\{q_{i}\} \given \theta} 
\end{equation}
where $\{ q_{i} \}$, again, represents the estimated group assignments.
Therefore $\widehat{\theta}$ can also be found by maximising the partition function defined in \cref{eq:partitionFunctionGroupAssignments} over $\theta$.
By minimising this function for $\theta$ results in stationarity conditions known as the \emphT{Nishimori conditions} in statistical physics \cite{DKM+13}.
There is an iterative method for learning the parameters based on the Nishimori conditions and can be written in terms of messages to be used in a BP algorithm
\begin{equation}
	\label{eq:inferenceParametersBPMessage1}
	n_{a} = \frac{1}{n} \sum_{i} b_{a}^{i}
\end{equation}
\begin{equation}
	\label{eq:inferenceParametersBPMessage2}
	c_{ab} = \frac{1}{n_{a} n_{b} n} \sum_{(i,j) \in E} \frac{c_{ab} (m_{a}^{i \rightarrow j}m_{b}^{j \rightarrow i} + m_{b}^{i \rightarrow j}m_{a}^{j \rightarrow i})}{Z^{ij}}
\end{equation}
where we denote the set of edges of $\graphvar{G}$ by $E$ and denote $Z^{ij}$ as the BP estimate for the partition function defined by
\begin{equation}
	\label{eq:partitionFunctionBPEstimate}
	Z^{ij} = \sum_{a<b} c_{ab} (m_{a}^{i \rightarrow j}m_{b}^{j \rightarrow i} + m_{b}^{i \rightarrow j}m_{a}^{j \rightarrow i}) + \sum_{a} c_{aa} m_{a}^{i \rightarrow j} m_{a}^{j \rightarrow i} \text{ for } (i,j) \in E
\end{equation}
The EM algorithm uses the above BP update equations for the expectation step starting with random initialisation for $\theta$.
Combining the parameter learning steps with the inference of node assignments enables a complete BP algorithm, which Decelle et al. have described in \cite{DKM+13} and called `\textsc{BP-LEARNING}'.

For the purposes of our testing using the synthetically generated data from the planted partition model, we shall use the C++ implementation of this algorithm provided by Decelle et al. \cite{DKM+13} which is available from \cite{ModeNet}.

%-----------------------------------------------------
%   NLPI and AMP Algorithms Section
%-----------------------------------------------------

\section{NLPI and AMP Algorithms}
\label{sec:NLPIAndAMPAlgorithmsCommunityDetection}

The following algorithms aim to partition networks based upon the hidden clique model described in \cref{subsec:hiddenCliqueModel}, in order to identify the underlying hidden community.

%-------------------------------------------
%   Non-linear Power Iteration Sub Section
%-------------------------------------------

\subsection{Non-linear Power Iteration}
\label{subsec:nonLinearPowerIteration}

The intuition behind the non-linear power iteration (NLPI) method is fairly straightforward.
Recall we wish to reconstruct the node assignment vector denoted by $\vecvar{u}$. \Cref{def:hcmNormlaisedAdjacencyMatrixDecomposed} shows $\vecvar{u}$ is the principal eigenvector of a rank-1 matrix in noise, called the normalised adjacency matrix, and denoted by $\widetilde{\matvar{A}}$.
We use the standard power iteration algorithm with one extra step; we additionally apply a separable non-linear function that acts component-wise.
We choose the non-linear function to force the reconstructed vector to adhere to one of the properties desired. In particular we can apply `positive-part thresholding' \cite{Mon13}, where we keep only the positive elements of the vector (and set the negative elements to zero) at each iteration.
The following is a recursive definition of one iteration of the general approach, where $t$ indexes the iteration
\begin{equation}
	\label{eq:NLPIAlgorithm}
	\begin{split}
		&\vecvar{z}^{t+1} = \widetilde{\matvar{A}} \widehat{\vecvar{u}}^{t} \\
		&\widehat{\vecvar{u}}^{t} = f_{t}(\vecvar{z}^{t})
	\end{split}
\end{equation}
where
\begin{equation}
	\label{eq:NLPIAlgorithmConditions}
	\begin{split}
		&\widehat{\vecvar{u}}^{0} = \transpose{[1,\dots,1]} \\
		&\vecvar{z} = \transpose{[z_{1},\dots,z_{n}]} \\
		&f_{t}(\vecvar{z}) = \transpose{[f_{t}(z_{1}),\dots,f_{t}(z_{n})]}
	\end{split}
\end{equation}
Since we will consider positive-part thresholding,
\begin{equation}
	\label{eq:NLPIAlgorithmThresholding}
	f_{t}(z_{i}) =
	\begin{cases}
		z_{i}& \text{if } z_{i} > 0\\
		0 & \text{otherwise}
	\end{cases}
\end{equation}
for all $i=1,\dots,n$.

This algorithm takes advantage of the fact that the leading eigenvector is non-negative, an observation identified in \cref{subsec:hiddenCliqueModel}.
We will test this algorithm based on positive-part thresholding empirically for synthetically generated networks for varying SNR and sizes of the hidden community in \cref{sec:NLPIAndAMPAlgorithms}.
However, it is also important to analyse the algorithm theoretically, for instance by quantifying the (possible) improvement over spectral methods in different regimes.
Unfortunately, analysing this algorithm in terms of precise asymptotics is very difficult since there are dependencies existent after any number of iterations \cite{Mon13}.

%-------------------------------------------
%   Approximate Message Passing Sub Section
%-------------------------------------------

\subsection{Approximate Message Passing}
\label{subsec:approximateMessagePassing}

Although the NLPI method works well in practice (and we shall show this empirically later), we still seek an algorithm that can also be analysed theoretically. The \emphT{approximate message passing} (AMP) algorithm involves one modification to the NLPI, where a memory term is subtracted.
The following is a recursive definition of one iteration of the general approach, where $t$ again indexes the iteration
\begin{equation}
	\label{eq:AMPAlgorithm}
	\begin{split}
		&\vecvar{z}^{t+1} = \widetilde{\matvar{A}} \widehat{\vecvar{u}}^{t} - b_{t}\widehat{\vecvar{u}}^{t-1}\\
		&\widehat{\vecvar{u}}^{t} = f_{t}(\vecvar{z}^{t})
	\end{split}
\end{equation}
where we define
\begin{equation}
	\label{eq:AMPAlgorithmConditions}
	\begin{split}
		&\widehat{\vecvar{u}}^{-1} = \transpose{[0,\dots,0]} \\
		&\widehat{\vecvar{u}}^{0} = \transpose{[1,\dots,1]} \\
		&\vecvar{z} = \transpose{[z_{1},\dots,z_{n}]} \\
		&f_{t}(\vecvar{z}) = \transpose{[f_{t}(z_{1}),\dots,f_{t}(z_{n})]} \\
		&b_{t} \equiv \frac{1}{n}\sum_{i=1}^{n}f_{t}^{'}(z_{i})
	\end{split}
\end{equation}
and again consider positive-part thresholding.
We remark that the explicit formula for $b_{t}$ is chosen since it cancels the statistical bias (i.e. decorrelates) on $\widehat{\vecvar{u}}^{t+1}_{i}$ due to $\widehat{\vecvar{u}}^{\leq t}_{i}$.
The explanation for this result is beyond the scope of this report, but we refer the reader to \cite{DMM09,MDM10,BM11,Mon11,BKS13} for details.


% Experiments on Synthetic Data
% Chapter 4

\chapter{Experiments on Synthetic Data}

\label{cha:experimentsOnSyntheticData}

%----------------------------------------------------------------------------------------

In this chapter we aim to experiment with community detection algorithms on synthetically generated data.
We shall consider each algorithm described in the previous chapter, and experiment with data created from the appropriate generative model.
The goal of the experiments on synthetic data is to understand how the underlying
network structure, and the variation of parameters therein, affects the performance of different algorithms \cite{RLK12}.
In general the experiments will proceed as follows. We generate a network with the appropriate block model and specified parameters, with an underlying ground-truth node assignments. We then measure the accuracy of the specific algorithm investigated as we vary model parameters. This then allows us to draw conclusions regarding the relative performance of community detection algorithms in controlled conditions given by networks with common properties.
We conclude by discussing the advantages and issues with all the algorithms investigated and provide recommendations for their use in certain circumstances.

%-----------------------------------------------------
%   NLPI and AMP Algorithms Section
%-----------------------------------------------------

\section{NLPI and AMP Algorithms}
\label{sec:NLPIAndAMPAlgorithms}

We shall use the procedure outlined in \cite{Mon13} to test the NLPI and AMP algorithms, where the synthetic data used is simply an instance of a normalised adjacency matrix. Recall \cref{def:hcmNormlaisedAdjacencyMatrixDecomposed}, where we decomposed the matrix into signal (weighted by a signal-to-noise ratio term) plus noise.
We will consider the case of a network generated by the hidden clique model with $n$ nodes and one hidden community consisting of $k$ nodes. We may choose the community memberships arbitrarily. Using previous notation, let us denote $\vecvar{u}$ as the indicator variable for nodes belonging to the hidden community (i.e. so that precisely $k$ elements of $\vecvar{u}$ are equal to one and the rest equal zero), $lambda$ as the SNR and $\matvar{Z}$ as the noise term.
We set the elements of $\matvar{Z}$ to be $Z_{ij} \sim \normal{0,1/n}$ i.i.d. entries. Therefore given a value of $\lambda$ and our ground-truth community assignments, we can construct the normalised adjacency matrix, once more denoted by $\widetilde{\matvar{A}}$, using \cref{def:hcmNormlaisedAdjacencyMatrixDecomposed}.
This matrix will serve as input to the NLPI and AMP algorithms, which produce an output representing their reconstructed node assignments, which we shall denote by $\widehat{\vecvar{u}}$.
We shall represent the accuracy of the algorithms by the inner product of the ground-truth and reconstructed node assignments, $\innerP{\vecvar{u},\widehat{\vecvar{u}}}$.
Notice this value lies between 0 and 1, and a larger number indicating improved reconstruction and better accuracy.

Now, with the framework set up, we can construct our own tests, seeking the accuracy as both the SNR and the size of the hidden community are varied.
Define a new variable $\varepsilon \equiv k/n$ representing the proportion of all nodes that belong to the hidden community, then we can construct a grid of points representing different values of $\lambda$ and $\varepsilon$.
We, finally, plot the values of $\innerP{\vecvar{u},\widehat{\vecvar{u}}}$ for each point on the grid to analyse the behaviour.

Utilising this approach with $n=500$ nodes, varying $\lambda$ between 0 and 1.2, choosing a grid resolution of 100 and running 50 iterations of the NLPI algorithm, we show in \cref{fig:NLPIAccuracyPlot}, the accuracy of the NLPI algorithm plotted against $\lambda$ and $\varepsilon$.

%---   FIGURE
\begin{figure}
	\centering
	\includegraphics[width=0.9\linewidth]{figures/NLPIMontanariSyntheticDataScalarProducts.png}
	\caption[Plot of accuracy for NLPI algorithm]{\label{fig:NLPIAccuracyPlot} A plot of $\innerP{\vecvar{u},\widehat{\vecvar{u}}}$ for the NLPI algorithm for different values of $\lambda$ and $\varepsilon$. The synthetic data is generated from a network with $n=500$ nodes and a grid with resolution 100, whilst the algorithm is run for 50 iterations.}
\end{figure}

The motivation for the NLPI algorithm is to detect the node assignments more accurately than standard PCA for small-sized hidden communities. The results can be analysed by focusing on the accuracy for small values of $\varepsilon$ and all ranges of $\lambda$.
We have previously discussed that the transition for standard PCA algorithms for small-size hidden communities is $\lambda = 1$, where, for values above this threshold, reconstruction is possible, and below, it is impossible.
\Cref{fig:NLPIAccuracyPlot} illustrates that for small values of $\varepsilon$ (e.g. between 0 and 0.2), the reconstructed vector is correlated with ground-truth (i.e. $\innerP{\vecvar{u},\widehat{\vecvar{u}}} > 0$).
In particular, we can get good reconstruction for values of $\lambda$ all the way down to 0.8. Thus we have shown to beat the spectral threshold!

Although the NLPI algorithm can be shown to beat the spectral threshold empirically by using the synthetic data strategy of \cite{Mon13}  that we used above, as we have already noted, analysing its asymptotic behaviour mathematically is not trivial.
This served as the motivation for the AMP algorithm where precise asymptotics could be derived theoretically (however this is beyond the scope of this report).
We shall now analyse the performance of the AMP algorithm using identical synthetic data (i.e. the same normalised adjacency matrix input for every value pair $(\lambda,\varepsilon)$ in the grid) as we tested the NLPI algorithm with. We also chose to run the AMP algorithms for 50 iterations.
We have plotted the accuracy for the AMP algorithm in \cref{fig:AMPAccuracyPlot}.

%---   ERROR
%%% need to add picture of (correct) AMP algorithm accuracy 

%---   FIGURE
\begin{figure}
	\centering
	\includegraphics[width=0.9\linewidth]{figures/AMPMontanariSyntheticDataScalarProducts.png}
	\caption[Plot of accuracy for AMP algorithm]{\label{fig:AMPAccuracyPlot} A plot of $\innerP{\vecvar{u},\widehat{\vecvar{u}}}$ for the AMP algorithm for different values of $\lambda$ and $\varepsilon$. The synthetic data is generated from the same network with $n=500$ nodes as used in \cref{fig:NLPIAccuracyPlot}, and a grid with resolution 100, whilst the algorithm is run for 50 iterations.}
\end{figure}

\Cref{fig:AMPAccuracyPlot} illustrates that for small values of $\varepsilon$ (e.g. between 0 and 0.2), the reconstructed vector is correlated with ground-truth.
In particular, we get good reconstruction for values of $\lambda$ all the way down to 0.8. Thus we have shown to beat the spectral threshold once more!
We also note that the striking similarity between \cref{fig:NLPIAccuracyPlot} and \cref{fig:AMPAccuracyPlot} is to be expected given the formulations of these algorithms.

We have seen how NLPI and AMP algorithms can be used to detect hidden communities in networks generated by a hidden clique model.
In particular we have seen the empirical improvements over standard spectral methods such as PCA.
Moreover, the NLPI approach of applying a suitable non-linear function as an extra step to traditional power-iteration algorithm may be used for any problem where an eigenvector with special properties (e.g. sparsity or non-negativity) needs to be found.
Although the application is beyond the scope of this report, the AMP algorithm has had much success when applied to \emphT{compressed sensing}; we refer the reader to \cite{DMM09,MDM10,BM11,Mon11,BKS13} for more details. 

% Community detection in financial networks
% Chapter 5

\chapter{Community Detection in Financial Networks}

\label{cha:communityDetectionFinancialNetworks}

%----------------------------------------------------------------------------------------

In this chapter we aim to apply algorithms to detect communities within real-world financial networks.
Firstly, we explain the process of gathering the data to create the financial networks.
Then we investigate and apply different community detection algorithms and compare their performance on both synthetically generated and the real-world data.

%-----------------------------------------------------
%   Financial Data Processing Section
%-----------------------------------------------------

\section{Constructing the Real-world Financial Network}
\label{sec:realWorldFinancialNetwork}

The dataset we use consists of daily closing prices of 80 stocks in the FTSE 100 index, which we obtained from \cite{YahFi}.
The time period considered is between the beginning of 2004 to the end of 2013, a total of 2501 prices. 
This is the data we obtained after the removal of a few data points due to incomplete data across different stocks.The complete list of stocks is given in \cref{app:listFTSE100Stocks}.
We then calculated, for each stock and for each time period, the logarithmic return. We generated a time series of these returns and associated each stock with a single time series.
By using the method described in \cref{subsec:financialNetworksConstructionBackground}, we proceeded to construct financial network represented by a fully-connected, undirected and weighted graph. There are 80 nodes in this network (each one representing one of the stocks), and the weights on the edges connecting any two nodes is the cross-correlation between the time series of returns associated with the stocks represented by the two nodes.
We stress, at this point, the data of the whole period (01/01/2004 - 01/01/2013) is currently represented by one single network.
\Cref{fig:expectedReturnsAndVolatility} shows a plot of the expected return against the volatility for each stock considered during this period.

%---   FIGURE
\begin{figure}
	\centering
	\includegraphics[width=0.7\linewidth]{figures/expectedReturnsAndVolatility.png}
	\caption[Plot of expected return against volatility for 80 FTSE 100 stocks.]{\label{fig:expectedReturnsAndVolatility} A plot of expected return against volatility of the 80 stocks in the FTSE 100 index considered. Data points obtained from price data during the period 01/01/2004 - 01/01/2013.}
\end{figure}

%-------------------------------------------
%   Random Matrix Theory Sub Section
%-------------------------------------------

\subsection{Random Matrix Theory}
\label{subsec:randomMatrixTheory}

The correlation matrix is representing the weighted adjacency matrix of the network, and in order to better understand the weights in the network, we wish to refer to an important result from Random Matrix Theory (RMT) that has been outlined in \cite{SM99,PGR+99,PBL05,MG13}. In particular, we wish to distinguish between random and non-random properties of empirical correlation matrices.

A correlation matrix created from $n$ random time series of length $T$, in the limits $n \rightarrow +\infty$ and $T \rightarrow +\infty$ with $1 < T/n < +\infty$, has a specific distribution of eigenvalues known as the \emphT{Sengupta-Mitra distribution} \cite{SM99,PBL05,FPW+11,MG13}. This distribution is defined by 
\begin{equation}
	\label{def:senguptaMitraDistribution}
	\rho(\lambda) =	
	\begin{cases}
		\frac{T}{n}\frac{\sqrt{(\lambda_{+} - \lambda)(\lambda - \lambda_{-})}}{2\pi\lambda}& \text{if } \lambda_{-} \leq \lambda \leq \lambda_{+} \\
		0 & \text{otherwise}
	\end{cases}
\end{equation}
where the maximum and minimum eigenvalues ($\lambda_{+}$ and $\lambda_{-}$ respectively) are given by
\begin{equation}
\label{eq:maxEigenvalueSM}
	\lambda_{+} = \left(1+\sqrt{\frac{n}{T}}\right)^{2}
\end{equation}
and
\begin{equation}
\label{eq:minEigenvalueSM}
	\lambda_{-} = \left(1-\sqrt{\frac{n}{T}}\right)^{2}
\end{equation}
Therefore the set of eigenvalues of an empirical correlation matrix that lies within this distribution is considered to occur purely as a result of random noise \cite{PBL05,FPW+11,MG13}. Moreover, we may regard any eigenvalue larger than $\lambda_{+}$ to represent important structure within the data \cite{PBL05,FPW+11,MG13}.

Analysing the deviation of the eigenvalue spectrum of empirical correlation matrices constructed from real-world financial data from the RMT distribution constitutes an effective method to filter noise out from the data.
For example, we constructed the correlation matrix from the FTSE 100 data set (described in \cref{sec:realWorldFinancialNetwork}) and plotted the eigenvalue spectrum for this matrix alongside the corresponding Sengupta-Mistra distribution (i.e. RMT prediction with $n = 80$ and $T = 2501$) in \cref{fig:eigenvalueSpectra}.
We observe two interesting regions of the eigenvalue spectrum outside the RMT prediction. Firstly, the largest eigenvalue of the correlation matrix, which we shall denote by $\lambda_{m}$, is much larger than all other eigenvalues. Also, the eigenvector associated with the largest eigenvalue, denoted by $v_{m}$, has all elements positive. This has been observed in many previous studies of empirical correlation matrices, and this eigenvalue is also called the \emphT{market mode} \cite{FPW+11,MG13}, meaning this component acts as a common factor influencing all assets within the market \cite{MG13}. Secondly, we observe a few eigenvalues just outside the RMT predicted region (i.e. eigenvalues just larger than $\lambda_{+}$ and much smaller than $\lambda_{m}$. We believe these components reflect a mesoscopic level of groups of stocks within the market (i.e. neither at the level of individual stocks in the form of noise, nor at the level of the entire market in the form of the market mode eigenvalue) \cite{MG13}, hence we expect members of these groups of stocks to demonstrate similar underlying properties, such as related sector classifications.

%---   FIGURE
\begin{figure}
	\centering
	\begin{subfigure}{.5\textwidth}
		\centering
		\includegraphics[width=0.8\linewidth]{figures/eigenvalueSpectra_n_80_T_2501.png}
		\caption{}
		\label{fig:eigenvalueSpectrumOriginal}
	\end{subfigure}%
	\begin{subfigure}{.5\textwidth}
		\centering
		\includegraphics[width=0.8\linewidth]{figures/eigenvalueSpectraZoomed_n_80_T_2501.png}
		\caption{}
		\label{fig:eigenvalueSpectrumZoomed}
	\end{subfigure}
	\caption[Plots of empirical and RMT predicted eigenvalue spectrum]{\label{fig:eigenvalueSpectra} Plots of eigenvalue spectra for empirical correlation matrix and RMT prediction, \subref{fig:eigenvalueSpectrumOriginal}, in addition to a zoomed-in version, \subref{fig:eigenvalueSpectrumZoomed}. The empirical correlation matrix was constructed from the daily log-returns of the FTSE 100 data set, and its eigenvalue spectrum is plotted in red. The RMT prediction is the Sengupta-Mitra distribution with appropriate parameters ($n = 80$, $T = 2501$), and is plotted in blue. The zoomed-in graph identifies the existence of eigenvalues outside of the region predicted by RMT, whilst the zoomed-out graph clearly shows the maximum eigenvalue (i.e the market mode eigenvalue) with a value of about 28.}
\end{figure}

We proceed to utilise the eigenvalue spectrum observed for the data set and the RMT prediction to filter out the empirical correlation matrix to reflect a mesoscopic structure, as achieved by \cite{MG13}. Recall the correlation matrix for our FTSE 100 data set is a $80 \times 80$ matrix denoted by $\matvar{C}$ and that we denote $\lambda_{i}$ as the $i-th$ eigenvalue of $\matvar{C}$ and $\vecvar{v_{i}}$ represents the eigenvector associated with $\lambda_{i}$. We are able to decompose this matrix as the sum of three matrices
\begin{equation}
\label{eq:decompositionCorrelationMatrix}
	\matvar{C} = \matvar{C}^{(r)} + \matvar{C}^{(g)} + \matvar{C}^{(m)}
\end{equation}
where $\matvar{C}^{(r)}$ represents the correlation matrix corresponding to the random components, defined by
\begin{equation}
\label{eq:randomCorrelationMatrix}
	\matvar{C}^{(r)} \equiv \sum_{i:\lambda_{i}\leq\lambda_{+}} \lambda_{i} \vecvar{v_{i}} \transpose{\vecvar{v_{i}}}
\end{equation}
$\matvar{C}^{(m)}$ represents the correlation matrix corresponding to the market mode component, defined by
\begin{equation}
\label{eq:marketModeCorrelationMatrix}
	\matvar{C}^{(m)} \equiv \lambda_{m} \vecvar{v_{m}} \transpose{\vecvar{v_{m}}}
\end{equation}
and $\matvar{C}^{(g)}$ represents the remaining correlations
\begin{equation}
\label{eq:remainingCorrelationMatrix}
	\matvar{C}^{(g)} \equiv \sum_{i:\lambda_{+} < \lambda_{i} < \lambda_{m}} \lambda_{i} \vecvar{v_{i}} \transpose{\vecvar{v_{i}}}
\end{equation}

Therefore, we now have a representation of a filtered empirical correlation matrix, $\matvar{C}^{(g)}$, which represents the mesoscopic (group level) correlations of the stocks, which we shall use, crucially, as the input to several community detection algorithms (it can be thought of as a weighted adjacency matrix of a new filtered network).

%-----------------------------------------------------
%   Community Detection Algorithms Section
%-----------------------------------------------------

\section{Community Detection Algorithms}
\label{sec:communityDetectionAlgorithms}

So far we have been able to construct a financial network based on the correlations of daily log returns of stocks and, using RMT, a filtered correlation matrix that represents a new financial network with links (hopefully) representing group-level correlation.
Although, the question still remains, given either the initial or filtered correlation matrix, how does one produce a set of groups of stocks with greater correlations within a group than between groups?
From previous sections, we understand the notion of community detection within graphs, which we shall also refer to as \emphT{binary networks}, and have analysed several algorithms that tackle this problem. However, in these problems we analysed adjacency matrices that contained binary elements (i.e. a `1' if an edge exists in the graph and a`0' otherwise), whereas, in this problem, we study a weighted adjacency matrix with elements as real numbers.
The reader should also note the adjacency matrix studied in previous sections is directly related to the structure of the network in question, whereas in this case, it is related to the weights of links between nodes.
This suggests the possibility of having to modify previously studied algorithms for this scenario.

Given the conclusions of our analysis in \cref{cha:communityDetectionAlgorithms}, and that the real-world data in this case is not well represented by generative random models, we shall use the modularity optimisation as a basis of some of our algorithms.

%-------------------------------------------
%   Naive Modularity Methods Sub Section
%-------------------------------------------

\subsection{Modularity Optimisation Methods}
\label{subsec:modularityOptimisationMethods}

Recall in \cref{sec:modularityBasedOptimisation}, we introduced the notion of modularity optimisation as a method for community detection within networks. We shall once more consider algorithms for modularity optimisation, but for the case of financial networks.

Let us denote a partition of $n$ nodes, in the financial network, into communities/groups by the vector $\vecvar{\sigma} = \transpose{[\sigma_{1},\dots,\sigma_{n}]}$, where $\sigma_{i}$ denotes the group to which node $i$ belongs to. This is essentially the group to which stock $i$ belongs. We then define the modularity for this partition, $Q(\vecvar{\sigma})$, by
\begin{equation}
\label{eq:modularityFinancialNetworks}
	Q(\vecvar{\sigma}) = \frac{1}{2m} \sum_{ij} \left(A_{ij} - \frac{k_{i}k_{j}}{2m} \right) \delta(\sigma_{i},\sigma_{j})
\end{equation}
where $\matvar{A}$ is the weighted adjacency matrix of the network, $k_{i} \equiv \sum_{j} A_{ij}$ and $2m \equiv \sum_{ij} A_{ij}$.
Note that in the case of binary networks, $A_{ij}$ represented the presence or absence of an edge between nodes $i$ and $j$, $d_{i}$ represented the degree of node $i$ and $m$ represented the total number of edges in the networks.
Since we are interested in financial networks, a naive approach would be to use the empirical correlation matrix, denoted by $\matvar{C}$, as the networks weighted adjacency matrix. Note that we are essentially ignoring the results of \cref{subsec:randomMatrixTheory}, and we will see in \cref{subsec:modifiedModularityOptimisationMethod} why this is not a good idea, both intuitively and mathematically.

For now, though, let us use the following relationship
\begin{equation}
\label{eq:weightedAdjacencyMatrixFinancialNetworks}
	A_{ij} = \frac{1}{2} \big( C_{ij}+1 \big) - \delta(i,j)
\end{equation}
where $\matvar{C}$ denotes the correlation matrix, and $\delta(i,j)$ removes self edges. From this definition we simply note that $A_{ij} \in [0,1]$.

We now focus on finding the partition, $\vecvar{\sigma}$, that maximises the modularity. We notice that a larger value of $A_{ij}$ implies larger correlation between the stocks $i$ and $j$, whilst a smaller value implies a lower correlation.
 Recalling how, for the case with binary networks, we sought after denser connections within groups and sparser connections between groups, we realise the modularity maximisation algorithms used on binary networks should have the same effect on financial networks.
This is intuitive since, in both types of networks, we aim to find the partition which maximises the sum of correlations or number of edges between nodes within same community and minimises the sum of correlations or number of edges between nodes belonging to different communities.

We shall select two familiar approaches, widely known in the literature, as algorithms for the modularity maximisation. Firstly we consider a greedy agglomerative method discussed earlier in the report, and secondly a method that uses spectral relaxation.

We consider the same greedy agglomerative method described in \cref{subsec:greedyAlgorithm}, which we noted is also applicable in this case of weighted networks.
The implementation code for this algorithm is in MATLAB and has been taken from \cite{ELM}.

The approach using the spectral relaxation can be summarised using the argument from \cite{DM}. Let $\matvar{B}$ denote the modularity matrix, defined by
\begin{equation}
\label{eq:modularityMatrixFinancialNetworks}
	B_{ij} = A_{ij} - \frac{d_{i}d_{j}}{2m}
\end{equation}
where $A_{ij}$, $d_{i}$ and $m$ are defined as before.
Also denote the set of nodes belonging to group $a$ by $\setvar{S}_{a} \equiv \{i:\sigma_{i} = a\}$.
The algorithm iterates by attempting to split the node members of a single group in an optimal fashion by using modularity.
Assume at one iteration there are $q$ groups so the partition is indexed by $[q] \equiv \{1,2,\dots,q\}$.
For some $a \in [q]$, let $\matvar{B}_{a}$ denote the submatrix restricted to nodes in $\setvar{S}_{a}$.
Let $\vecvar{v} \in \realsR^{\cardinality{\setvar{S}_{a}}}$ denote the sign vector given the algorithm operating on $\matvar{B}_{a}$. Then the change in modularity is given by
\begin{equation}
\label{eq:changeModularityFinancialNetworks}
	\begin{split}
		&\Delta Q = \frac{1}{2m} \left( \sum_{i,j \in \setvar{S}_{a}} B_{ij}\frac{(1+v_{i}v_{j})}{2} - \sum_{i,j \in \setvar{S}_{a}} B_{ij} \right) \\
		&= \frac{1}{4m} \left( \sum_{i,j \in \setvar{S}_{a}} B_{ij}v_{i}v_{j} - \sum_{i,j \in \setvar{S}_{a}} B_{ij} \right)  \\
		&= \frac{1}{4m} \transpose{\vecvar{v}} \widetilde{\matvar{B}}_{a} \vecvar{v}
	\end{split}
\end{equation}
where $\widetilde{\matvar{B}}_{a} = \matvar{B}_{a} - diag \left( \sum_{j \in \setvar{S}_{a}} B_{ij} \right)$.

The algorithm accepts the splitting of the group which maximises the modularity difference, and terminates when reaches a threshold regarding the size of the groups and the possible improvement in modularity at a given iteration.
We have implemented this algorithm of \cite{DM} using MATLAB, stressing the input is defined by \cref{eq:weightedAdjacencyMatrixFinancialNetworks}.

%-------------------------------------------
%   Modified Modularity Method Sub Section
%-------------------------------------------

\subsection{Modified Modularity Optimisation Method}
\label{subsec:modifiedModularityOptimisationMethod}

We move on to understand the potential issues with the naive approach hinted in the previous section. Intuitively, the term $d_{i}d_{j}/2m$ reflects the null hypothesis that the observed network structure is wholly based on the degrees of the nodes.
This idea is sound when applied to binary networks, however, these terms do not have a specific meaning when applied to financial networks since we are interested in the correlation matrix rather than the structure of the underlying graph.
The quantities $d_{i} \equiv \sum_{j} C_{ij}$ and $2m \equiv \sum_{ij} C_{ij}$ do not make up a clear notion of a null model when combined.
Moreover, \cite{MG13} have shown the naive approach is mathematically incorrect and leads to biased results.
In order to detect communities within the correlation matrix that leads to correct results, we seek improvement in the construction of the null model for the network.
We have already discussed, in \cref{subsec:randomMatrixTheory}, how to construct a filtered correlation matrix that takes into account the random noise at the level of individual stocks and a market-wide component.
Therefore, this filtered matrix, which we have denoted by $\matvar{C}^{(g)}$ and defined in \cref{eq:remainingCorrelationMatrix}, can be used as a modularity matrix, since it reflects a suitable null model (i.e. random noise plus market wide correlation) subtracted from the observed correlation matrix.

Our aim, here, is to use $\matvar{C}^{(g)}$ as input to an algorithm that can detect communities by maximising modularity. We hope this provides better results than the naive modularity methods discussed previously. The construction of an algorithm is not quite as straightforward, though, since we take into account the modification of the modularity matrix.

We must first, though, introduce a new formulation of modularity given a partition, as used by \cite{MG13}, which we denote by $Q_{n}(\vecvar{\sigma})$
\begin{equation}
\label{eq:newModularityFinancialNetworks}
	Q_{n}(\vecvar{\sigma}) = \frac{1}{C_{norm}} \sum_{ij} C^{(g)}_{ij} \delta \left( \sigma_{i},\sigma_{j} \right).
\end{equation}
where $C_{norm}$ is a normalisation term defined by 
\begin{equation}
\label{eq:newModularityNormalisationConstantFinancialNetworks}
	C_{norm} \equiv \sum_{ij} \abs{C_{ij}}
\end{equation}
which just ensures the value of the newly-defined modularity lies within the interval $[-1,+1]$.
The new formulation of modularity is specifically aimed to detect mesoscopic-level communities.

The `modified' modularity method is a spectral clustering algorithm based on the exact same technique considered for binary networks considered in \cref{sec:spectralClustering}, but instead we shall use the filtered correlation matrix, $\matvar{C}^{(g)}$, as input.
We simply find the eigenvectors of $\matvar{C}^{(g)}$, use them to construct the embedded vectors, which are then clustered using the familiar k-means clustering algorithm (same procedure as with binary networks).
We use different values for the number of groups (within an appropriate range), then run the algorithm for each one and choose the partition with the best value for the re-defined modularity (\cref{eq:newModularityFinancialNetworks}).

%---   ERROR
%%% need to describe modified greedy agglomerative method for modified modularity maximisation


%-----------------------------------------------------
%   Synthetic Data Testing Section
%-----------------------------------------------------

\section{Synthetic Data Testing}
\label{sec:syntheticDataTesting}

Before applying the four algorithms discussed previously to our financial network based on the FTSE 100 data set, we must run tests to confirm we can correctly detect correlated sets of time series in synthetically generated benchmark cases, as also outlined by \cite{MG13}.
We consider a benchmark data set of correlation matrices with $100$ time series (i.e. we consider $n = 100$ stocks in the data set) divided into $10$ communities of $10$ correlated time series. Note the length of the time series is chosen to be $T=2500$ to reflect similar conditions to the real data set and is also prescribed by RMT (i.e. $T>n$).
The set consists of correlation matrices generated with different levels of correlations between the groups and within a group reflecting different signal-to-noise ratios (SNR). We set the ground-truth partition, which we denote by $\vecvar{\sigma}^{*}$, to be the same across all correlation matrices.
We considered $6$ such benchmarks within the correlation matrix set with varying SNR, and have illustrated $2$ such examples in \cref{fig:benchmarkCorrelationMatrices}.
We confirmed that, in all the benchmarks, all four methods succeeded in identifying $\vecvar{\sigma}^{*}$.

%---   FIGURE
\begin{figure}
	\centering
	\begin{subfigure}{.5\textwidth}
		\centering
		\includegraphics[width=0.8\linewidth]{figures/syntheticCorrelationMatrices_1.png}
		\caption{}
		\label{fig:benchmarkCorrelationMatrix1}
	\end{subfigure}%
	\begin{subfigure}{.5\textwidth}
		\centering
		\includegraphics[width=0.8\linewidth]{figures/syntheticCorrelationMatrices_3.png}
		\caption{}
		\label{fig:benchmarkCorrelationMatrix3}
	\end{subfigure}
	\caption[Plots of synthetically generated benchmark correlation matrices]{\label{fig:benchmarkCorrelationMatrices} Plots of synthetically generated benchmark correlation matrices, one with low SNR (a value of 0.5), \subref{fig:benchmarkCorrelationMatrix1}, in addition to one with higher SNR (a value of 1), \subref{fig:benchmarkCorrelationMatrix3}. These are two examples from a set created from $100$ time series divided into $10$ communities of $10$ correlation time series. The blocks along the diagonal represent cross-correlations between members of the same group, and are thus high (i.e. close to one), whereas off-diagonal blocks represent cross-correlations between time series belonging to different communities reflecting noise (system-wide or additional inter-community correlations).}
\end{figure}

%-----------------------------------------------------
%   Application to Real-world Financial Network Section
%-----------------------------------------------------

\section{Application to Real-world Financial Network}
\label{sec:applicationToRealWorldFinancialNetwork}

We now apply the three modularity optimisation methods discussed in \cref{sec:communityDetectionAlgorithms} to the FTSE 100 data collected.
Our intentions are two fold.
Firstly we wish to understand the community structure of the stocks, in particular with regard to the composition of each community based upon the industry sectors of the stocks.
Secondly, we wish to build a comparative analysis of the three different algorithms based on empirical data.

We firstly begin by looking at the communities generated by the three modularity methods discussed in \cref{sec:communityDetectionAlgorithms}, using all of the data.
\Cref{fig:outputFTSE100} displays the communities identified by all the algorithms by grouping the tickers of the stocks based upon label colouring.
We see from \cref{fig:outputCommunitiesGreedy,fig:outputCommunitiesSpectralRelaxation} that both traditional modularity methods, the greedy algorithm and the spectral relaxation method respectively, have identified two distinct communities; whilst from \cref{fig:outputCommunitiesSpectralClustering}, we notice the modified modularity spectral clustering technique has identified four communities.
This is the outcome we expected given the null model assumed discounted both random and market-wide correlations \cite{MG13}, meaning the spectral clustering algorithm was successful in finding previously undetected communities.

%---   FIGURE
\begin{figure}
\centering
	\begin{subfigure}{.5\textwidth}
		\centering
		\includegraphics[width=0.9\linewidth]{figures/outputFTSE100FastNewmanMethod.png}
		\caption{}
		\label{fig:outputCommunitiesGreedy}
	\end{subfigure}%
	\begin{subfigure}{.5\textwidth}
		\centering
		\includegraphics[width=0.9\linewidth]{figures/outputFTSE100MontanariMethod.png}
		\caption{}
		\label{fig:outputCommunitiesSpectralRelaxation}
	\end{subfigure}\\
	\begin{subfigure}{.5\textwidth}
		\centering
		\includegraphics[width=0.9\linewidth]{figures/outputFTSE100FinancialSpectralClusteringMethod.png}
		\caption{}
		\label{fig:outputCommunitiesSpectralClustering}
	\end{subfigure}
	\caption[Communities of the FTSE 100 data generated using three different algorithms.]{\label{fig:outputFTSE100} Communities of the FTSE 100 data generated using three different algorithms. The name of each label represents the associated stock's ticker (see \cref{app:listFTSE100Stocks}) whilst the colour of the label represents community memberships. The resulting communities after the greedy algorithm was applied is shown in \subref{fig:outputCommunitiesGreedy}, where there were two communities generated. The resulting communities after the algorithm based on a spectral relaxation was applied is shown in \subref{fig:outputCommunitiesSpectralRelaxation}, where there were also two communities generated. The resulting communities after the spectral clustering algorithm based on the modified modularity matrix was applied is shown in \subref{fig:outputCommunitiesSpectralClustering}, where there were four communities generated.}
\end{figure}

A first look at the tickers of stocks belonging to each community does not help to uncover any particular patterns.
Therefore we analyse the relative composition of each community identified based on the industry sector of each stock as shown in \cref{fig:outputFTSE100PieCharts}.
The industry sectors for each stock we consider is given in \cref{app:listFTSE100Stocks}.
Each community is represented by a single pie chart (labelled by letter), where the colour legend defined in \cref{tab:outputFTSE100PieChartsColourLegend} indicates the industry sector association.
We note that the actual sizes of the pie charts do not provide any information, only the fractions of the different coloured regions are important.

%---   TABLE
\begin{table}
	\caption{Colour representation for 10 industry sectors used to classify FTSE 100 stocks, to be used as a legend.}
	\label{tab:outputFTSE100PieChartsColourLegend}
	\centering
	\includegraphics[width=.6\linewidth]{figures/outputFTSE100PieChartsColourLegend.png}
\end{table}

%---   FIGURE
\begin{figure}
\centering
	\begin{subfigure}{.5\textwidth}
		\centering
		\includegraphics[width=0.9\linewidth]{figures/outputFTSE100IndustriesPieChartsFastNewmanMethod.png}
		\caption{}
		\label{fig:outputPieChartsCommunitiesGreedy}
	\end{subfigure}%
	\begin{subfigure}{.5\textwidth}
		\centering
		\includegraphics[width=0.9\linewidth]{figures/outputFTSE100IndustriesPieChartsMontanariMethod.png}
		\caption{}
		\label{fig:outputPieChartsCommunitiesSpectralRelaxation}
	\end{subfigure}\\
	\begin{subfigure}{.5\textwidth}
		\centering
		\includegraphics[width=0.9\linewidth]{figures/outputFTSE100IndustriesPieChartsFinancialSpectralClusteringMethod.png}
		\caption{}
		\label{fig:outputPieChartsCommunitiesSpectralClustering}
	\end{subfigure}
	\caption[Pie charts showing the relative composition of each generated community based on industry sectors of the FTSE 100 stocks for three different algorithms.]{\label{fig:outputFTSE100PieCharts} Pie charts showing the relative composition of each generate community based on the industry sectors of the stocks (see \cref{app:listFTSE100Stocks}). Individual communities are labelled by letter, where each community is represented by one pie chart, and the colour legend for all these pie charts is shown in \cref{tab:outputFTSE100PieChartsColourLegend}. The result for communities generated by the greedy algorithm is shown in \subref{fig:outputPieChartsCommunitiesGreedy}, whilst the result for communities generated by the spectral relaxation method is shown in \subref{fig:outputPieChartsCommunitiesSpectralRelaxation}. The output for communities generated by the modified modularity spectral clustering method is show in \subref{fig:outputPieChartsCommunitiesSpectralClustering}.}
\end{figure}

A few interesting observations can be made.
Certain industry sectors tend to subjugate their respective communities.
For example, stocks belonging to the Utilities, Basic Materials and Financials sectors seem to have remained correlated over the time period between 2004 and 2013.
But there also examples of sectors that are split amongst different communities.
A similar outcome has also been observed in \cite{MG13} with a data set consisting of stocks from the S\&P 500 index.
This suggests that subgroups of stocks within a sector are often more correlated with stock from different top-level sectors than from their own.
Unfortunately, we were not able to obtain sub-classifications for industry sectors due to lack of information available, however, we believe it would be beneficial to extract this form of qualitative information and combine with our quantitative approach identify interesting patterns in correlations.

We now introduce the notion of a renormalised filtered correlation between two communities $A$ and $B$ considered in \cite{MG13}.
We consider the renormalised filtered correlation matrix, denoted by $\widetilde{\matvar{C}}^{(g)}$, as a matrix whose dimension equal the number of communities detected with each element defined by
\begin{equation}
\label{eq:renormalisedFilteredCorrelationMatrix}
	\widetilde{C}^{(g)}_{AB} \equiv \sum_{i \in A} \sum_{j \in B} C^{(g)}_{ij}
\end{equation}
We shall use the renormalised filtered correlation to test whether different communities, generated are mutually less correlated than expected under the null model \cite{MG13}.
\Cref{fig:renormalisedFilteredCorrelationMatrices} plots the renormalised filtered correlation matrices for the communities generated by three different algorithms.

%---   FIGURE
\begin{figure}
\centering
	\begin{subfigure}{.5\textwidth}
		\centering
		\includegraphics[width=0.6\linewidth]{figures/renormalisedFilteredCorrelationMatrixFastNewmanMethod.png}
		\caption{}
		\label{fig:renormalisedFilteredCorrelationMatrixGreedyAlgorithm}
	\end{subfigure}%
	\begin{subfigure}{.5\textwidth}
		\centering
		\includegraphics[width=0.6\linewidth]{figures/renormalisedFilteredCorrelationMatrixMontanariMethod.png}
		\caption{}
		\label{fig:renormalisedFilteredCorrelationMatrixSpectralRelaxation}
	\end{subfigure}\\
	\begin{subfigure}{.5\textwidth}
		\centering
		\includegraphics[width=0.6\linewidth]{figures/renormalisedFilteredCorrelationMatrixFinancialSpectralClusteringMethod.png}
		\caption{}
		\label{fig:renormalisedFilteredCorrelationMatrixSpectralClustering}
	\end{subfigure}
	\caption[Plots of renormalised filtered correlation matrices for the communities generated by three different algorithms.]{\label{fig:renormalisedFilteredCorrelationMatrices} Plots of renormalised filtered correlation matrices for the communities generated by three different algorithms. Each heat map represents the cross-correlations between the communities identified by each algorithm. Each off-diagonal element is calculated as the sum of the correlations between the corresponding pair of communities, whilst the diagonal entries represents the sum of correlations of nodes belonging to the specific community. The filtered correlation matrix for the output of the greedy algorithm is shown in \subref{fig:renormalisedFilteredCorrelationMatrixGreedyAlgorithm}, whilst the result for the communities generated by the spectral relaxation method is shown in \subref{fig:renormalisedFilteredCorrelationMatrixSpectralRelaxation}. The output for communities generated by the modified modularity spectral clustering method is show in \subref{fig:renormalisedFilteredCorrelationMatrixSpectralClustering}.}
\end{figure}

We see that all algorithms identify partitions that, on aggregate, are anti-correlated with one another using the definition of renormalised filtered correlation.

We conclude that introducing the notion of modified modularity has helped to detect more communities at the mesoscopic scale by applying a spectral clustering algorithm.
The properties of these communities are similar, and the partitions are anti-correlated with one another.
The main benefit of applying the greedy algorithm, though, is that it converges faster to the resulting node assignments.
A step forward would be, therefore, to introduce a greedy agglomerative method based on modified modularity, and indeed take advantage of both the removal of noise at the level of individual stocks and the fast nature of the greedy algorithm.
Overall, as discussed in \cref{subsec:portfolioTheoryBackground}, this property is very useful for investors following mean-variance portfolio theory, since it provides a basis for picking stocks belonging to mesoscopic-level communities that are, on average over a specified period of time (decided by the investor and given by selective use of available data), negatively correlated with one another.



% Temporal Evolution in financial networks
% Chapter 6

\chapter{Temporal Evolution of Financial Networks}

\label{cha:temporalEvolutionFinancialNetworks}

%----------------------------------------------------------------------------------------

In this chapter we aim to understand the temporal evolution of a financial network with respect to the evolving correlation structure of the constituting assets.
In \cref{cha:communityDetectionFinancialNetworks} we considered a static financial network, where the weight of an edge represents the correlation coefficient between the two time series associated with the nodes connected by the edge, considering the entire time period to construct one financial network.
We then applied community detection techniques to uncover groups of stocks from the FTSE 100 that were correlated more than a null hypothesis suggests.
An issue with this approach is the static nature of the network, since investors wish to understand the strength of correlations in price movements in order to dynamically manage investment risk in their portfolios \cite{FPW+11}.
We build on the results given in \cref{cha:communityDetectionFinancialNetworks} and investigate community dynamics utilising time-dependent correlation structures with application to the same FTSE 100 data set, complementing the work of \cite{OCK+02,OKK03,BD10,FPM+10,FPW+11}.
This approach enables us to identify major changes in the underlying financial market, and the same ideas may be applied to other financial markets and asset classes (by use of other available data sets), underlining the potential utility of the techniques considered.

%-----------------------------------------------------
%   Financial Data Processing Section
%-----------------------------------------------------

\section{Constructing Time-evolving Financial Networks}
\label{sec:timeEvolvingFinancialNetwork}

Once more we consider the same FTSE 100 data set used earlier in the report, but we move away from using the data of the whole period to construct one single network.
Instead, we examine the data for several, overlapping, time windows that collectively cover the whole period.
We generate one financial network for each time window in the following way, also used by \cite{OCK+02,OKK03,BD10,FPM+10,FPW+11}.
Recall from \cref{subsec:financialNetworksConstructionBackground} each node in the network (i.e. asset) is associated with a single time series consisting of the daily logarithmic return.
We now let the number of time steps considered, $T$, equal the length of each time window rather than the length of the whole time period.
Proceeding to create a correlation matrix based upon the standardised time series, as before, we have developed one correlation matrix for each time window.
As previously, each correlation coefficient (i.e. entry in the correlation matrix) between any two time series is the weight of the edge connecting the nodes associated with the ties series in the network.
We have created a sequence financial networks by rolling the time windows of returns through the whole data set, with each network describing the underlying correlation structure of the market at a particular time interval.
By shifting the time window there is an overlap in data in the consecutive windows, but this allows us to track the evolution of correlations and identify changes in the behaviour of the market at many different times during the whole time period \cite{FPW+11}.
This is particularly consequential given that the data set (01/01/2004 - 11/11/2013) includes some of the most significant periods for the state of developed economics worldwide in recent timed, a point we will address later on in this section.

Immediately, though, there is the requirement of deciding upon both the window length and the length of the overlap (i.e. the duration of time any one window has data overlapped with the preceding time window).
Any particular choice of $T$ is a compromise between overly noisy and overly smoothed correlation coefficients \cite{OCK+02,FPW+11}.
We study the distribution of correlation coefficients for different choices of the parameters to decide on appropriate values.
For instance, \cref{fig:distributionCorrelationCoefficientsRollover1} compares the distribution of the correlation coefficients for three different values of the time window length, $T=100$, $150$ and $200$ (days), whilst fixing the overlap period to 1 (day).
Here we plot the mean, variance and skewness to characterise the distribution and realise these signals are quite noisy.

%---   FIGURE
\begin{figure}
	\centering
	\includegraphics[width=0.8\linewidth]{figures/distributionCorrelationCoefficientsRollover_1.png}
	\caption[Plots characterising the distribution of correlation coefficients for a fixed roll-over period of 1 day and varying window lengths.]{\label{fig:distributionCorrelationCoefficientsRollover1} Plots characterising the distribution of correlation coefficients over time for a fixed roll-over period of 1 day and varying window lengths, $T=100$, $150$ and $200$. The top graph plots the mean value, the centre graph plots the variance whilst the bottom graph plots the skewness all against the whole time period for the FTSE 100 data set.}
\end{figure}

Whereas, \cref{fig:distributionCorrelationCoefficientsRollover10} compares the distribution of the correlation coefficients for three different values of the time window length, $T=100$, $150$ and $200$ (days), whilst fixing the overlap period to 10 (days).
Once more we plot the mean, variance and skewness to characterise the distribution, but now realise these signals are smoother.

%---   FIGURE
\begin{figure}
	\centering
	\includegraphics[width=0.8\linewidth]{figures/distributionCorrelationCoefficientsRollover_10.png}
	\caption[Plots characterising the distribution of correlation coefficients for a fixed roll-over period of 10 days and varying window lengths.]{\label{fig:distributionCorrelationCoefficientsRollover10} Plots characterising the distribution of correlation coefficients over time for a fixed roll-over period of 10 days and varying window lengths, $T=100$, $150$ and $200$. The top graph plots the mean value, the centre graph plots the variance whilst the bottom graph plots the skewness all against the whole time period for the FTSE 100 data set.}
\end{figure}

We decide to trade-off the over-smoothing of shorter window lengths with the longer overlap period, and proceed with $T=100$ and a value of overlap period equal to 10 days.
These choices results, for our data, in 240 correlation matrices (and hence financial networks) spanning the period between 01/01/2004 and 11/11/2013.
This completes the construction of time evolving financial networks, which we can now use as an empirical test basis for our proceeding analysis.

%-----------------------------------------------------
%   Temporal Evolution of Correlation Section
%-----------------------------------------------------

\section{Temporal Evolution of Asset Correlation}
\label{sec:temporalEvolutionAssetCorrelation}

We have seen in \cref{subsec:randomMatrixTheory}, how the eigenvalue spectrum of the empirical correlation matrix generated by the aggregate at set indicated correlations not compatible with the null model of a combination of random and market-wide components.
This result became a justification for seeking mesoscopic level communities by discovering patterns in the correlation structure of the financial network, and was a building block of our modified modularity approach.
We now wish to investigate the temporal evolution of the correlations between the stocks by analysing the behaviour of the respective correlation matrix eigenvalues and eigenvectors at each time window.
Having previously used tools from RMT, we now utilise a closely linked \cite{FPW+11} and widely used technique, known as principal component analysis (PCA).
PCA is a statistical technique that uses an orthogonal transformation to represent the covariance structure of a set of variables through a smaller number of linear combinations of these variables. \cite{Jol02,FPW+11,Gil14}.

We shall consider the following derivation based on \cite{Jol02,UIO03,FPM+10,FPW+11,Gil14}.
Firstly, recall $T$ represents the time window length and $n$ represents the number of assets in the network (these values equal to 100 and 80, respectively, in our specific empirical example).
Let us denote the matrix with entries consisting of the standardised logarithmic returns, for any one particular time window, by a $n \times T$ matrix $\matvar{X}$.
The empirical correlation matrix, also equal to the covariance matrix of $\matvar{X}$, is denoted by $\matvar{C}$ and defined by
\begin{equation}
	\label{eq:correlationMatrix}
	\matvar{C} = \frac{1}{T} \matvar{X} \transpose{\matvar{X}}
\end{equation}
where each element $C_{ij}$ represents the cross-correlation between time series $i$ and $j$.
The matrix $\matvar{X}$ constitutes the set of observed (original) variables, and the PCA method seeks to find a linear transformation, denoted by the matrix, $\matvar{\Omega}$, that maps $\matvar{X}$ into a set of uncorrelated variables given by $\matvar{Y}$.
$\matvar{Y}$ is an $n \times T$ matrix defined by
\begin{equation}
	\label{eq:uncorrelatedVariables}
	\matvar{Y} = \matvar{\Omega} \matvar{X} 
\end{equation}
where each row $\vecvar{y}_{i}$ (for $i =1,\dots,n$) corresponds to a principal component (PC) of $\matvar{X}$.
The first row of the matrix $\matvar{\Omega}$, denoted by $\vecvar{\omega}_{1}$, is selected so the first PC, $\vecvar{y}_{1}$, is aligned with the direction of maximal variance in the $n$-dimensional space defined by $\matvar{X}$.
Each subsequent PC accounts for the maximal variance of $\matvar{X}$ subject to the condition that the vectors $\vecvar{\omega}_{j}$ are mutually orthonormal.
This implies that
\begin{equation}
	\label{eq:PCCoefficients}
	 \vecvar{\omega}_{j} \transpose{\vecvar{\omega}_{k}} =
	\begin{cases}
		1 & \text{if } j = k\\
		0 & \text{otherwise}
	\end{cases}
\end{equation}
for all $j,k = 1,\dots,n$.
The correlation matrix is an $n \times n$ symmetric and diagonalisable matrix, which can be written as 
\begin{equation}
	\label{eq:diagonalisationCorrelationMatrix}
	\matvar{C} = \matvar{\Phi} \matvar{\Lambda} \transpose{\matvar{\Phi}}
\end{equation}
where $\matvar{\Phi}$ is an orthogonal matrix of its eigenvectors and $\matvar{\Lambda}$ is diagonal matrix consisting of the associated eigenvalues.
From the result that the eigenvectors of the covariance matrix correspond to the directions of maximal variance \cite{Jol02}, $\matvar{\Omega} = \transpose{\matvar{\Phi}}$, and thus we can determine the PCs using the decomposition given by \cref{eq:diagonalisationCorrelationMatrix}.

\Cref{subsec:randomMatrixTheory} compares the eigenvalue spectra of an empirical correlation matrix (using our entire data set) against a correlation matrix created from $n$ random time series of length $T$ in the limiting case.
The analysis found significant features of the spectra of the empirical correlation matrix.
Most of the eigenvalues were contained in a region explained as random noise and given by the Sengupta-Mitra distribution.
However, a selection of eigenvalues lay outside this region, as illustrated in \cref{fig:eigenvalueSpectra}, suggesting some form of correlation structure between the microscopic and macroscopic exists.
We crucially note that the condition of $T/n > 1$ must be satisfied for the result to hold, but only requires appropriate selection of the parameter $T$.
We have achieved in satisfying this constraint and thus the result from RMT applies to all of our time-windowed financial networks as well.

We can now combine the results from PCA and RMT to study the temporal evolution of correlation structure, considering the approaches from \cite{UIO03,FPM+10,FPW+11}.
We denote the covariance matrix for the PC matrix $\matvar{Y}$ by $\matvar{\Sigma}$, which is defined as
\begin{equation}
	\label{eq:PCCovarainceMatrix}
	\matvar{\Sigma} = \frac{1}{T} \matvar{Y} \transpose{\matvar{Y}}
\end{equation}
Using \cref{eq:correlationMatrix,eq:uncorrelatedVariables,eq:diagonalisationCorrelationMatrix}, we can re-write $\matvar{\Sigma}$ as
\begin{equation}
	\label{eq:PCCovarainceMatrixRewritten}
	\matvar{\Sigma} = \frac{1}{T} \matvar{\Omega} \matvar{X} \transpose{\matvar{X}} \transpose{\matvar{\Omega}} = \matvar{\Omega} \matvar{C} \transpose{\matvar{\Omega}} = \transpose{\matvar{\Phi}} \matvar{C} \matvar{\Phi} = \matvar{\Lambda}
\end{equation}
We wish to find the total variance in the logarithmic returns for all assets.
Let us denote the $i$-th entry along the diagonal of $\matvar{\Lambda}$ by $\lambda_{i}$ (i.e. $\lambda_{i} = \Lambda_{ii}$).
Then the total variance for $\matvar{X}$ is given by
\begin{equation}
	\label{eq:totalVarianceLogarithmicReturns}
	\trace{\matvar{C}} = \sum_{i=1}^{n} \lambda_{i} = \trace{\matvar{\Lambda}} = n
\end{equation}
Therefore the proportion of the total variance in $\matvar{X}$ given by the $i$-th PC is given by
\begin{equation}
	\label{eq:proportionVarianceLogarithmicReturns}
	\frac{\Sigma_{ii}}{\trace{\matvar{C}}} = \frac{\lambda_{i}}{n}
\end{equation}
We can now analyse the time-varying nature of the proportion of variance of returns explained by certain PCs.

%---   FIGURE
\begin{figure}
	\centering
	\includegraphics[width=0.8\linewidth]{figures/eigenvalueContributionsPC.png}
	\caption[Plots of the contribution of the top 5 principal components to the total variance in returns against time.]{\label{fig:eigenvalueContributionsPC} Plots of the contribution of the top 5 principal components to the total variance in returns against time. Each data point corresponds to the term $\lambda_{i}/n$, defined in \cref{eq:proportionVarianceLogarithmicReturns}, for $i=1,\dots,5$ and a particular time window.}
\end{figure}

\Cref{fig:eigenvalueContributionsPC} shows the proportion of variance of returns explained by the top 5 PCs for each time window, so we plot $\lambda_{i}/n$ for $i=1,\dots,5$ and every time-windowed financial network.
The fraction of the variance explained by the first PC increased between 2005 and towards the end of 2006, then dipped for a few months through towards the middle of 2007.
There was a sharp rise beginning at the middle of 2007, around the time when the United States subprime mortgage industry collapsed \cite{GrWik}.
Also several central banks stepped in with lending to the interbank lending market in August 2007 in order to prevent a liquidity crisis \cite{GrWik}.
The bursting of the United States housing bubble had a major impact on the health of major worldwide financial institutions due to their massive exposure to mortgage-backed securities on their balance sheets \cite{GrWik}.
For instance, Merrill Lynch was taken over by Bank of America and Lehman Brother filed for bankruptcy on 15/09/2008, and we see the proportion of variance of returns contributed by the first PC increased from this date onwards towards 2011, a significant period of time during the recession \cite{FPW+11,GrWik}.
The large contribution in variance of returns by only one principal component indicates a significant amount of common variation in the market (i.e. in the FTSE 100 exchange in our case) \cite{FPW+11}.
In other words, the market, as a whole, became more correlated during these years, in particular a very distinct reaction to a major crisis to a few internaitonal financial institutions.
This information implied by {fig:eigenvalueContributionsPC} relates to the intuition that the performance of stocks should decline, as a whole, during a financial crisis, so then the returns will be more highly correlated during this period.
In addition, in 2004, the top twelve PCs contributed to 64.12\% of the variance of returns, whilst, in 2009, just the top six PCs contributed to 62.82\%.
The fact that fewer components are required to account for a similar amount of variation in returns, in 2009, compared to five years earlier suggests an increase in the common variation between stocks in the exchange.
Moreover, it also implies the correlation structure of this marker can be explained by many fewer factors than $n$ sets of asset time series.
The last fact supports the eigenvalue spectra analysis of \cref{subsec:randomMatrixTheory}.


% Conclusion and Future Work
% Chapter 7

\chapter{Conclusion and Future Work}

\label{cha:Conclusion}

%----------------------------------------------------------------------------------------

Within this report we have studied the topic of community detection in networks, with two principal contributions outlined, involving the exploration of algorithms and their application to both controlled environments (realised by synthetically-generated data) and empirical data (representing real-world financial networks).

Firstly, we analysed a range of community detection algorithms, with different techniques underpinning them, that are present in the literature, and tested them on synthetic data.
By undertaking this approach, through the generation of networks with community structure using well-known statistical block models, we have studied the performance of these algorithms on networks with varying properties.
This enabled us to advocate the use of certain algorithms depending on the properties of the underlying network we wish to detect communities from.
Particular highlights include the belief propagation algorithm, which performed better than all the other methods in terms of accuracy in the sparse regime and is more extensible to larger-scale sparse networks due to lower computational complexity.
We also noted that this BP algorithm was specifically designed to detect communities created from the statistical generative model, which is not well representative of many real-world networks.
On the other hand, the modularity optimisation methods studied, and specifically the greedy algorithms, have performed well on a range of real-world networks.

Our second task involved the application of community detection algorithms to real-world, time-evolving financial networks, where we partition the network to uncover isolated groups of assets that are, on aggregate, more highly correlated within groups than between groups.
We motivated this problem by discussing its practicality in mean-variance portfolio theory, where investors could start with a mesoscopic-level grouping of assets for selecting their portfolios.
Using an empirical data set, that we constructed from the prices of 80 stocks traded on the FTSE 100 exchange between 2004 and 2013, we generated a static financial network.
By modifying and tailoring modularity optimisation techniques, studied in the first part of the report, for this specific application, we identified communities within that were not detectable using naive modularity methods, indicating a notable improvement.
Moreover, we constructed time-evolving financial networks from the same data set, by using a time windowing procedure, to represent the temporal evolution of correlation structure of the assets. 
Investigating community detection in dynamic networks, by analysing Laplacian dynamics on multislice networks, led to a more general notion of modularity that is also applicable to time-evolving networks.
We focused on one particular algorithm that has been designed to optimise the generalised modularity function, and is present in the literature, known as the generalised Louvain method.
By tailoring the inputs to this method, based on our notion of modified modularity (that was successfully applied in the static network regime), we applied this method which enabled us to observe the dynamic community structure of FTSE 100 stocks in a turbulent period for financial markets worldwide.

%-----------------------------------------------------
%   Future Work Section
%-----------------------------------------------------

\section{Future Work}
\label{sec:futureWork}

We outline several ideas for potential future research directions that we have discovered through our work during the project.
We divide our recommendations for future work into two areas; the first is a more general outlook on community detection algorithms, and the second focuses on the financial networks application specifically.

We would like to make two general points on the topic of community detection in networks.
Firstly, there is a lack of a widely accepted prescription defining a community within a network.
There are several statistical models which generate networks reflecting community structure that one finds intuitive, but I believe there is a lack of consensus on an appropriate benchmark within the research community.
Determining such a benchmark enables the provision of a specification which any community detection algorithm can be judged against.
Secondly, as the reader may have noticed, we did not consider overlapping community structure at any time during the project, instead we only focused on non-overlapping communities (i.e. each node may only belong to one community).
However, we believe the community structure of real-world networks, and in particular social networks, is better represented by modelling through overlapping communities.
The reason why we did not focus on this, and the potential research questions involve a better definition of the concept of overlapping communities and the introduction of reliable statistical models which can generate graphs with overlapping community structure.

There are two ways to extend our approach of community detection to financial networks.
The first simply involves gathering more empirical financial data sets, both across different asset classes and different worldwide markets.
By applying the methods on more data sets, we can provide a deeper understanding of community structure in financial markets worldwide not just the stocks on the FTSE 100 exchange.
Examples include price data on equity, bond, commodity and foreign exchange markets.
A more exploratory direction is the design of dynamic community detection algorithms with better empirical computational complexity and run time than the generalised Louvain method, since with more data and assets to capture in the network, this will become a problem.
We understand this is a big step forward since there are not many reliable dynamic community detection algorithms present in the literature (it currently seems challenging enough to find appropriate methods in the static network case), and the generalised Louvain method is probably the best, but it would represent a huge achievement with important applications across many disciplines and empirical data sets.

%----------------------------------------------------------------------------------------
%	THESIS CONTENT - APPENDICES
%----------------------------------------------------------------------------------------

\appendix % Cue to tell LaTeX that the following 'chapters' are Appendices

% List of FTSE 100 Stocks considered
% Appendix A

\chapter{List of Stocks}

\label{app:listFTSE100Stocks}

%---   TABLE
% Table generated by Excel2LaTeX from sheet 'Sheet2'
\begin{center}
\begin{longtable}{|l|l|l|}
	\caption[List of FTSE 100 Stocks studied]{The ticker symbol, name and industry of the 80 FTSE 100 companies studied. Price data and classifications were obtained from \cite{IcWik,LSE,YahFi}.}
	\label{tab:listFTSE100Stocks} \\
	\hline \multicolumn{1}{|c|}{\textbf{Ticker}} & \multicolumn{1}{c|}{\textbf{Name}} & \multicolumn{1}{c|}{\textbf{Industry}} \\ \hline 
	\endfirsthead
	\hline \hline
	\endlastfoot
    AAL   & Anglo American & Basic Materials \\
    ABF   & Associated British Foods & Consumer Goods \\
    ADN   & Aberdeen Asset Management & Financials \\
    AGK   & Aggreko & Industrials \\
    AHT   & Ashtead Group & Industrials \\
    AMEC  & AMEC  & Basic Materials \\
    ANTO  & Antofagasta & Basic Materials \\
    ARM   & ARM Holdings & Technology \\
    AV    & Aviva & Financials \\
    AZN   & AstraZeneca & Health Care \\
    BAB   & Babcock International Group & Industrials \\
    BARC  & Barclays & Financials \\
    BATS  & British American Tobacco & Consumer Goods \\
    BG    & BG Group & Basic Materials \\
    BLND  & British Land Company & Technology \\
    BLT   & BHP Billiton & Basic Materials \\
    BNZL  & Bunzl & Industrials \\
    BP    & BP    & Basic Materials \\
    BRBY  & Burberry Group & Consumer Goods \\
    BSY   & British Sky Broadcasting Group & Consumer Services \\
    BT-A  & BT Group & Technology \\
    CCL   & Carnival & Consumer Goods \\
    CNA   & Centrica & Utilities \\
    CPI   & Capita & Industrials \\
    DGE   & Diageo & Consumer Goods \\
    EZJ   & easyJet & Consumer Services \\
    GFS   & G4S   & Industrials \\
    GKN   & GKN   & Consumer Goods \\
    GSK   & GlaxoSmithKline & Health Care \\
    HMSO  & Hammerson & Financials \\
    HSBA  & HSBC Holdings & Financials \\
    IAG   & International Consolidated Airlines Grp & Consumer Services \\
    IHG   & InterContinental Hotels Group & Consumer Services \\
    IMT   & Imperial Tobacco Group & Consumer Goods \\
    ITRK  & Intertek Group & Industrials \\
    ITV   & ITV   & Consumer Services \\
    JMAT  & Johnson Matthey & Basic Materials \\
    KGF   & Kingfisher & Consumer Services \\
    LAND  & Land Securities Group & Financials \\
    LGEN  & Legal \& General Group & Financials \\
    LLOY  & Lloyds Banking Group & Financials \\
    LSE   & London Stock Exchange Group & Financials \\
    MGGT  & Meggitt & Industrials \\
    MKS   & Marks and Spencer Group & Consumer Services \\
    MRO   & Marathon Oil Corporation & Oil \& Gas \\
    MRW   & Wm. Morrison Supermarkets & Consumer Services \\
    NG    & National Grid & Utilities \\
    NXT   & NEXT  & Consumer Services \\
    OML   & Old Mutual & Financials \\
    PRU   & Prudential & Financials \\
    PSN   & Persimmon & Industrials \\
    PSON  & Pearson & Consumer Services \\
    RB    & Reckitt Benckiser Group & Consumer Goods \\
    RBS   & Royal Bank of Scotland Group & Financials \\
    RDSB  & Royal Dutch Shell & Basic Materials \\
    REL   & Reed Elsevier & Consumer Services \\
    REX   & Rexam & Consumer Goods \\
    RIO   & Rio Tinto & Basic Materials \\
    RR    & Rolls-Royce Holding & Industrials \\
    RRS   & Randgold Resources Limited & Basic Materials \\
    SBRY  & J Sainsbury & Consumer Services \\
    SDR   & Schroders & Financials \\
    SGE   & The Sage Group & Technology \\
    SHP   & Shire & Health Care \\
    SMIN  & Smiths Group & Industrials \\
    SN    & Smith \& Nephew & Health Care \\
    SSE   & SSE   & Utilities \\
    STAN  & Standard Chartered & Financials \\
    SVT   & Severn Trent & Utilities \\
    TATE  & Tate \& Lyle & Consumer Goods \\
    TPK   & Travis Perkins & Industrials \\
    TSCO  & Tesco & Consumer Services \\
    ULVR  & Unilever & Consumer Goods \\
    UU    & United Utilities Group & Utilities \\
    VOD   & Vodafone Group & Technology \\
    WEIR  & The Weir Group & Industrials \\
    WMH   & William Hill & Consumer Services \\
    WOS   & Wolseley & Industrials \\
    WPP   & WPP ORD 10P & Consumer Services \\
    WTB   & Whitbread & Consumer Services \\
\end{longtable}%
\end{center}





\backmatter

%----------------------------------------------------------------------------------------
%	BIBLIOGRAPHY
%----------------------------------------------------------------------------------------

\label{Bibliography}

%%% Add bibliography here
\bibliographystyle{hieeetr.bst}
\bibliography{finalReportBibliography}

\end{document}  