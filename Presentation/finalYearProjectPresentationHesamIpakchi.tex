%%%%%%%%%%%%%%%%%%%%%%%%%%%%%%%%%%%%%%%%%
% Beamer Presentation
% LaTeX Template
% Version 1.0 (10/11/12)
%
% This template has been downloaded from:
% http://www.LaTeXTemplates.com
%
% License:
% CC BY-NC-SA 3.0 (http://creativecommons.org/licenses/by-nc-sa/3.0/)
%
%%%%%%%%%%%%%%%%%%%%%%%%%%%%%%%%%%%%%%%%%

%----------------------------------------------------------------------------------------
%	PACKAGES AND THEMES
%----------------------------------------------------------------------------------------

% uncomment to prevent displaying images/comment to display images
%\PassOptionsToPackage{draft}{graphicx}

\documentclass{beamer}

\mode<presentation> {

% The Beamer class comes with a number of default slide themes
% which change the colors and layouts of slides. Below this is a list
% of all the themes, uncomment each in turn to see what they look like.

%\usetheme{default}
%\usetheme{AnnArbor}
%\usetheme{Antibes}
%\usetheme{Bergen}
%\usetheme{Berkeley}
%\usetheme{Berlin}
%\usetheme{Boadilla}
%\usetheme{CambridgeUS}
%\usetheme{Copenhagen}
%\usetheme{Darmstadt}
%\usetheme{Dresden}
%\usetheme{Frankfurt}
%\usetheme{Goettingen}
%\usetheme{Hannover}
%\usetheme{Ilmenau}
%\usetheme{JuanLesPins}
%\usetheme{Luebeck}
\usetheme{Madrid}
%\usetheme{Malmoe}
%\usetheme{Marburg}
%\usetheme{Montpellier}
%\usetheme{PaloAlto}
%\usetheme{Pittsburgh}
%\usetheme{Rochester}
%\usetheme{Singapore}
%\usetheme{Szeged}
%\usetheme{Warsaw}

% As well as themes, the Beamer class has a number of color themes
% for any slide theme. Uncomment each of these in turn to see how it
% changes the colors of your current slide theme.

%\usecolortheme{albatross}
%\usecolortheme{beaver}
%\usecolortheme{beetle}
%\usecolortheme{crane}
%\usecolortheme{dolphin}
%\usecolortheme{dove}
%\usecolortheme{fly}
%\usecolortheme{lily}
%\usecolortheme{orchid}
%\usecolortheme{rose}
%\usecolortheme{seagull}
%\usecolortheme{seahorse}
%\usecolortheme{whale}
%\usecolortheme{wolverine}
\usecolortheme{sidebartab}

\setbeamertemplate{footline} % To remove the footer line in all slides uncomment this line
%\setbeamertemplate{footline}[page number] % To replace the footer line in all slides with a simple slide count uncomment this line

%\setbeamertemplate{navigation symbols}{} % To remove the navigation symbols from the bottom of all slides uncomment this line
}

\usepackage{graphicx} % Allows including images

\usepackage{booktabs} % Allows the use of \toprule, \midrule and \bottomrule in tables

%----------------------------------------------------------------------------------------
%	DEFINITIONS OF NEW SETTINGS
%----------------------------------------------------------------------------------------

\usepackage{subcaption}
\usepackage{tabularx}
\usepackage[nodayofweek]{datetime}

\graphicspath{ {./images/} }

\setbeamertemplate{itemize items}[triangle]
\setbeamertemplate{enumerate items}[default]

\newenvironment{variableblock}[3]{%
  \setbeamercolor{block title}{#2}
  \setbeamercolor{block body}{#3}
  \begin{block}{#1}}{\end{block}}

\usepackage{amssymb}% http://ctan.org/pkg/amssymb
\usepackage{pifont}% http://ctan.org/pkg/pifont
\newcommand{\cmark}{\ding{51}}%
\newcommand{\xmark}{\ding{55}}%

%----------------------------------------------------------------------------------------
%	TITLE PAGE
%----------------------------------------------------------------------------------------

\title[Community Detection in Networks]{Community Detection in Networks} % The short title appears at the bottom of every slide, the full title is only on the title page

\author{Hesam Ipakchi} % Your name
\institute[Imperial College London] % Your institution as it will appear on the bottom of every slide, may be shorthand to save space
{
Imperial College London \\ % Your institution for the title page
\medskip
\textit{hesam.ipakchi10@imperial.ac.uk} % Your email address
}
\newdate{presentationDate}{24}{06}{2014}
\date{\displaydate{presentationDate}} % Date, can be changed to a custom date

\begin{document}

\begin{frame}
\titlepage % Print the title page as the first slide
\end{frame}

%----------------------------------------------------------------------------------------
%	PRESENTATION SLIDES
%----------------------------------------------------------------------------------------

\begin{frame}
	\frametitle{Outline}
	\begin{itemize}
		\vfill\item Community structure in networks
		\begin{itemize}
			\item Intuitive Definition
			\item Example illustrations
		\end{itemize}
		\vfill\item Community detection algorithms
		\begin{itemize}
			\item Algorithms studied
			\item Synthetic data testing set-up
			\item Comparison using results from testing
		\end{itemize}
		\vfill\item Application on financial networks
		\begin{itemize}
			\item Motivation
			\item Application on a static network
			\item Extension to dynamic (time-evolving) networks
		\end{itemize}
	\end{itemize}
\end{frame}

%----------------------------------------------------------------------------------------

\begin{frame}
	\frametitle{Community structure in networks}
	\begin{itemize}
		\vfill\item Networks are represented by graphs, consisting of nodes and edges
		\vfill\item \textbf{\textcolor{blue}{Communities}} are groups of nodes with denser connections within groups and sparser connections between groups
		\vfill\item \textbf{\textcolor{blue}{Community detection}} algorithms involve partitioning the network into communities
		\vfill\item Many real-world applications including social networks, biological networks, financial networks...
	\end{itemize}
\end{frame}

%----------------------------------------------------------------------------------------

\begin{frame}
	\frametitle{Example illustrations}
	\begin{columns}[c]
		\column{.5\textwidth} % Left column and width
		\begin{figure}
			\centering
			\includegraphics[width=1.0\linewidth]{ppmExampleGraph.png}
		\end{figure}
		\column{.5\textwidth} % Right column and width
		\begin{figure}
			\centering
			\includegraphics[width=1.0\linewidth]{hcmExampleGraph.png}
		\end{figure}
	\end{columns}
\end{frame}

%----------------------------------------------------------------------------------------

\begin{frame}
	\frametitle{Community detection algorithms}
	Study a range of algorithms based on different techniques:
	\begin{itemize}
		\vfill\item Spectral clustering
		\vfill\item Modularity Optimisation
		\begin{itemize}
			\item Greedy algorithms
			\item Spectral algorithms
			\item Simulated annealing
		\end{itemize}
		\vfill\item Belief propagation
		\vfill\item Non-linear power iteration
	\end{itemize}
	\begin{variableblock}{}{}{bg=white,fg=red}
		\centering
		\textbf{NOT} considering massive data sets or overlapping communities!
	\end{variableblock}
\end{frame}

%----------------------------------------------------------------------------------------

\begin{frame}
	\frametitle{Statistical block models}
	Compare algorithms in a synthetic data testing framework
	\begin{itemize}
		\vfill\item Use statistical block models to generate random graphs which exhibit community structure
		\vfill\item Tweaking model parameters captures varying network properties (e.g. size, sparsity, number of communities, edge-occurrence probabilities)
		\vfill\item Provides theoretical setting to test and compare algorithms
		\vfill\item Two popular models: planted partition model, `hidden clique model'
	\end{itemize}
\end{frame}

%----------------------------------------------------------------------------------------

\begin{frame}
	\frametitle{Synthetic data testing set-up}
	For each community detection algorithm:
	\begin{itemize}
		\vfill\item Decide upon an appropriate generative model for this specific algorithm
		\vfill\item Construct synthetic data set by generating various networks with different underlying parameters of the model
		\vfill\item Apply the algorithm to each network in the data set and measure accuracy
	\end{itemize}
\end{frame}

%----------------------------------------------------------------------------------------

\begin{frame}
	\frametitle{Comparison of algorithms}
	\begin{table}
		\begin{tabular}{l l l}
			\toprule
			\textbf{Algorithm} & \textbf{Advantages} \textcolor{green}{\cmark} & \textbf{Disadvantages} \textcolor{red}{\xmark}\\
			\midrule
			Spectral Clustering & Simple & Accuracy $\downarrow$ as sparsity $\uparrow$ \\
			& & Quite slow \\
			& & Need \# communities \\
			& & as an input \\
			\midrule
			Greedy method & Simple & Accuracy $\downarrow$ as sparsity $\uparrow$ \\
			& Fast & \\
			& Works on larger graphs & \\
			\midrule
			Belief propagation & Very good accuracy & \\
			& Very fast (for sparse) & \\
			& Works on larger graphs & \\
			\bottomrule
		\end{tabular}
	\end{table}
	\begin{variableblock}{}{}{bg=white,fg=black}
		\centering
		\textcolor{red}{\textbf{Problem:}} generative models are not well representative of real-world networks!
	\end{variableblock}
\end{frame}

%----------------------------------------------------------------------------------------

\begin{frame}
	\frametitle{Financial Networks: Motivation}
	Consider portfolio selection problem for investor: how to select assets to form the `best' portfolio that aligns with risk and return preferences?
	\begin{itemize}
		\vfill\item Famous technique: \textcolor{blue}{\textbf{mean-variance portfolio theory}}
		\vfill\item \textcolor{blue}{\textbf{Idea:}} construct portfolio which generates the mean return desired but with lowest variance of all possible selections.
		\vfill\item Ideal case: make portfolio with lowest possible inter-asset correlations
		\vfill\item Use sample estimates of mean, variance and cross-correlation of asset returns from historical data
		\vfill\item Also beneficial for dynamic rebalancing of portfolio for risk management
	\end{itemize}
\end{frame}

%----------------------------------------------------------------------------------------

\begin{frame}
	\frametitle{Financial Networks: Construction}
	One possible approach: construct a undirected, fully connected and weighted graph!
	\begin{itemize}
		\vfill\item Each node represents an asset, with the (standardised) time series of returns for the asset associated with the node
		\vfill\item Weight of an edge connecting two nodes is the cross-correlation between the two time series' associated with those nodes (based on time averages)
		\vfill\item Weighted adjacency matrix of the graph is the empirical correlation matrix of the asset returns!
	\end{itemize}
	\textcolor{blue}{\textbf{Idea:}} communities in financial networks represent groups of assets whose aggregate average correlation is higher within groups and lower between groups \\
	$\therefore$ provide investors with (small number of) `baskets' of assets
\end{frame}

%----------------------------------------------------------------------------------------

\begin{frame}
	\frametitle{Our data set}
	\begin{itemize}
		\vfill\item Collected daily price data for 80 stocks traded on FTSE 100 exchange between 01/01/2004 and 11/11/2013
	\end{itemize}
	\begin{figure}
		\centering
		\begin{subfigure}{.5\textwidth}
			\centering
			\includegraphics[width=0.8\linewidth]{correlationMatrix_FTSE100_n_80_T_2501.png}
		\end{subfigure}%
		\begin{subfigure}{.5\textwidth}
			\centering
			\includegraphics[width=0.8\linewidth]{priceAndLogReturnTimeSeries.png}
		\end{subfigure}
	\end{figure}
\end{frame}

%----------------------------------------------------------------------------------------

\begin{frame}
	\frametitle{Community detection approach}
	Underlying technique: \textcolor{blue}{\textbf{modularity optimisation}}
	\begin{itemize}
		\vfill\item First consider two `naive' modularity maximisation methods: greedy algorithm and spectral relaxation
		\vfill\item Then consider two `modified' modularity methods, tailored for this specific application: spectral clustering and Louvain method
		\vfill\item \textcolor{blue}{\textbf{Result:}} able to detect finer-tuned and more communities using tailored approach $\implies$ higher quality partitions and notable improvement for adapted modularity techniques
	\end{itemize}
	\begin{variableblock}{}{}{bg=white,fg=black}
		\centering
		\textcolor{red}{\textbf{Problem:}} only considered one static network so far! Require \textcolor{blue}{\textbf{dynamic}} community detection to capture time-evolving correlation structure
	\end{variableblock}
\end{frame}

%----------------------------------------------------------------------------------------

\begin{frame}
	\frametitle{Extension to dynamic networks}
	Need to alter our approach:
	\begin{enumerate}
		\vfill\item Construct time-evolving networks from data set
		\begin{itemize}
			\vfill\item Use \textcolor{blue}{\textbf{time-windowing}} procedure to generate set of `network slices' and create one correlation matrix per time window
			\vfill\item For our data set: window size of 100, overlap length of 10 $\implies$ 240 correlation matrices
		\end{itemize}
		\vfill\item Apply dynamic community detection algorithm
		\begin{itemize}
			\vfill\item Consider the \textcolor{blue}{\textbf{generalised Louvain method}} previously applied to empirical neuroscience data in the literature
			\vfill\item Similar procedure to Louvain method but designed to optimise a generalised notion of modularity for dynamic networks
			\vfill\item Use modified modularity matrices as input for each network slice to tailor the method for our application
		\end{itemize}
	\end{enumerate}
\end{frame}

%----------------------------------------------------------------------------------------

\begin{frame}
	\frametitle{Evaluation}
	\begin{columns}[c]
		\column{.5\textwidth} % Left column and width
		\begin{itemize}
			\vfill\item Able to understand \textcolor{blue}{\textbf{temporal evolution of communities}} with smooth transitions in community memberships obtained
			\vfill\item Similar procedure is applicable for \textcolor{blue}{\textbf{different asset classes}} if data sets are collected
		\end{itemize}
		\column{.5\textwidth} % Right column and width
		\begin{figure}
			\centering
			\includegraphics[width=1.0\linewidth]{testMultiSliceLouvainMethodCommunities_omega_1.png}
		\end{figure}
	\end{columns}
	\begin{variableblock}{}{}{bg=white,fg=black}
		\centering
		\textcolor{red}{\textbf{But...}} the generalised Louvain method is slow and performance depends on appropriate parameter choice
	\end{variableblock}
\end{frame}

%----------------------------------------------------------------------------------------

\begin{frame}
	\frametitle{Future work}
	Some directions for research in this area...
	\begin{itemize}
		\vfill\item Lack of a consensus on precise definition of a `community' $\implies$ \textcolor{blue}{\textbf{no single benchmark}} to compare algorithms exists
		\vfill\item Little focus on \textcolor{blue}{\textbf{overlapping communities}} which could lead to better models for real-world networks
		\vfill\item Increase in availability of time-stamped network data sets enables the study and application of \textcolor{blue}{\textbf{dynamic community detection}} algorithms
		\vfill\item Significant improvements in \textcolor{blue}{\textbf{computational complexity}} enables partitioning of networks with up to millions of nodes, but algorithms are approximate methods and not very reliable
	\end{itemize}
\end{frame}

%----------------------------------------------------------------------------------------

\begin{frame}
	\frametitle{Summary}
	\begin{itemize}
		\vfill\item Introduced concept of community structure in networks and motivated use for community detection algorithms
		\vfill\item Mentioned several community detection algorithms and described the synthetic-data testing framework used to compare them
		\vfill\item Considered a real-world application of financial networks
		\vfill\item Identified time-evolving communities consisting of FTSE 100 stocks over the last decade
		\vfill\item Several interesting and important research directions exist
	\end{itemize}
\end{frame}

%----------------------------------------------------------------------------------------

\begin{frame}
	\Huge{\centerline{\textcolor{red}{Thanks for listening!}}}
	\Huge{\centerline{Any Questions?}}
\end{frame}

%----------------------------------------------------------------------------------------

\end{document}